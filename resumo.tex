\begin{abstract} %%%\section*{Resumo}

Este trabalho tem como objetivo apresentar, por meio de estudo de caso, os
n�veis de conformidade de seguran�a da informa��o que podem ser aplicados
por uma empresa conforme a norma \iso {---} parte 1. Sua realiza��o envolveu
pesquisa bibliogr�fica e de campo, as quais tiveram foco nos controles de
seguran�a de uma divis�o da industria farmac�utica e tem como desafio manter a
seguran�a das suas informa��es.


A pesquisa de campo realizada baseia-se nas recomenda��es e controles de
seguran�a da a norma \iso. Tamb�m foram consideradas informa��es de autores
especialistas na �rea de \SI para interpreta��o dos resultados obtidos.


� apresentado nesse trabalho um estudo de caso, o qual aborda a efic�cia do
controle de seguran�a de uma industria farmac�utica, foi desenvolvida, uma
pesquisa de campo para avaliar a conformidade dessa empresa com a norma
\iso. S�o apresentados tamb�m estudos sobre \SI, teoria geral de contratos e
aspectos de \SI { } na gest�o de contratos.

Com base no objetivo geral deste trabalho e nos resultados obtidos do estudo de
caso, foi poss�vel identificar atrav�s da pesquisa que a Farmac�utica A possui
todos os 10 dom�nios recomendados pela norma \iso, embora n�o apresente os
melhores padr�es recomendados.

\vspace{5cm}

\textbf{Palavras-Chave: contratos, seguran�a da informa��o, gest�o de contratos}
\end{abstract}
