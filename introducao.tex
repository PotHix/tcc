\section{Introdução}

A web tem crescido muito nos últimos anos, e suas novas ferramentas trazem várias
facilidades para desenvolver aplicações para a web, e isso também pode ser aplicado
a jogos casuais, trazendo também para os games uma facilidade de distribuição e portabilidade,
por utilizar apenas um browser moderno.

Mostrar as possibilidades de inovação e negócio que as novas tecnologias implementadas
nos navegadores modernos trazem para o mundo de desenvolvimento de jogos casuais,
comparando-as com as tecnologias existentes no mercado.

O trabalho vai apresentar os conceitos e utilizações das novas ferramentas utilizadas
nos navegadores modernos, focando principalmente em HTML5 e Javascript.
Será abordado a história do HTML, mostrando sua evolução com o passar do tempo, chegando
a seu estado atual, explorando quais a sua finalidade e quais os problemas que ele foi criado para resolver.
Logo em seguida será abordado a história do Javascript e sua evolução até seu estado
atual, mostrando qual a sua ligação com o HTML e como a sua evolução é importante
para o futuro da web, não somente com páginas comuns, mas com experimentos multimídia também.

Será apresentada também uma introdução do CSS e seu uso, para assim formar "o ciclo
do HTML5", que é formado por HTML, Javascript e CSS.

Com uma visão maior de HTML5, o trabalho vai focar em seus recursos e como isso pode
ser usado para o desenvolvimento de games. A ideia é descrever cada um de seus recursos
chave e mostrar passo a passo, com exemplos, onde ele pode ser aplicado e porque isso
é viável e útil, comparando com as tecnologias existentes no mercado atualmente. Para
a comparação serão avaliados alguns itens como, performance, dependências, portabilidade, recursos e viabilidade.

Com o grande crescimento da tecnologia dos navegadores, muita coisa se tornou possível
e viável, uma delas é o desenvolvimento de jogos casuais. Com a constante evolução
dos interpretadores de Javascript, HTML e CSS dos navegadores mais populares, é possível
desenvolver jogos simples e divertidos e conseguir um bom retorno com isso. Atualmente,
várias tecnologias são utilizadas para se fazer jogos para a web, cada uma com suas
vantagens e limitações, e o HTML5 (e seus complementos), entrou na competição, prometendo
se tornar o padrão para multimídia na web.

De acordo com \cite{flanagan2006javascript}, Javascript é uma linguagem de programação
com capacidades de Orientação a objetos comumente utilizada em
navegadores, sendo que, nesse contexto ela tem seu propósito extendido com objetos que permitem
aos scripts uma interação com o usuário, controlando o navegador e alterando o conteúdo
do documento que aparece na janela do navegador.
Utilizando a interação com o navegador que o Javascript proporciona, é possível movimentar
objetos pelo navegador, adicionar conteúdo, criar novos objetos, fazer desenhos, guardar
informações, entre outras coisas, bastando manipular o HTML da página.

HTML é uma linguagem de marcação de hipertexto que é utilizado como base para a internet
atual. O HTML mudou bastante desde sua versão inicial em 1991
\cite{powell2003html}, no início o HTML não era muito definido, e não tinha um padrão, até que que a
IETF (Internet Engeenering Task Force) começou a padronizá-lo em 1995, lançando a
primeira versão padronizada do HTML, que ficou conhecida como HTML 2.0.
Com o tempo o HTML foi se tornando mais padronizado, ganhando validações e novas integrações
(com CSS por exemplo), tornando-o mais versátil e adequado as novas necessidades da
web. Outra vantagem que a padronização trouxe, foi a facilidade de implementação pelos
navegadores, que agora implementavam com menos frequência o seu próprio estilo de marcação.
A ultima versão do HTML desenvolvida é o HTML5, que traz funcionalidades que dão um
olhar diferente para a internet, ou seja, ao invés de apenas se preocupar com texto
e formatação, o HTML5 visa atender as necessidades multimídia da internet, cobrindo
todos os espaços multimídia que tiveram que ser contornados com outras tecnologias,
como por exemplo vídeo, audio e desenho.
Com o crescimento do HTML, algumas coisas foram alteradas na parte de visualização,
removendo algumas tags que eram utilizadas para isso, como a tag font por exemplo.
Com a remoção dessas tags surgia a utilização de outra forma para
estilização de páginas, o CSS.

CSS é uma linguagem para criação de folhas de estilo, que interage com um documento
HTML, dando a ele uma melhor visualização, manipulando fontes, cores e espaçamentos.
A versão 2 do CSS se tornou padrão em maio de 1998, conforme
\cite{zeldman2009designing} (pg. 248), e atualmente está sendo
desenvolvida a versão 3, que já está parcialmente implementada nos navegadores modernos,
apesar de ainda estar em construção (assim como o HTML5).
Utilizando essas tecnologias emergentes da web é possível fazer jogos casuais simples
e divertidos. A definição de "jogos casuais" varia bastante conforme o autor, e este
trabalho vai utilizar a linha de pensamento utilizada pelo autor
\cite{trefry2010casual}. Uma das definições que ajuda a resumir um jogo casual é: "Um jogo casual não exige do jogador horas de
dedicação para cada seção de jogo, o jogo deve ser jogado aos poucos, em pequenas
porções, isso vai ajudar o jogo a se adequar a quantidade de tempo que os jogadores
de jogos casuais podem dedicar a ele.
A tecnologia que domina o mercado de jogos casuais atualmente é o flash, que devido
a sua grande adoção na época que a internet carecia de multimídia, se tornou "padrão"
em todos os computadores atuais, apesar de ser um software de terceiros instalado
como plugin para os navegadores. Java tem uma pequena porcentagem no total de jogos
na web, e é pouco usada ultimamente, diferentemente do framework Unity, que pode exportar
os games para a web, e tem crescido bastante desde seu lançamento.

O HTML5 traz a possibilidade de suprir várias funcionalidades de cada uma dessas tecnologias,
com a vantagem de ser um padrão implementado pelos empresas que desenvolvem os navegadores
mais conhecidos do mercado, e que já estão fazendo seus browsers para celulares e
video games, abrindo mais possibilidades para um jogo simples estar disponível em mais plataformas.
