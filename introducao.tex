\section{Introdução}

A web tem crescido muito nos últimos anos, e suas novas ferramentas trazem várias
facilidades para desenvolver aplicações para a mesma, e isso também pode ser aplicado
a jogos casuais, trazendo também para os games uma facilidade de distribuição e portabilidade,
por utilizar apenas de um navegador compatível com as novas tecnologias
para que o jogo seja executado.

%TODO: Enumerar as tres tecnologias como sendo tecnologias chave

Javascript é uma linguagem de programação com capacidades de orientação a objetos comumente utilizada em
navegadores \cite{flanagan2006javascript}, sendo que, nesse contexto ela tem seu propósito estendido com objetos que permitem
aos scripts uma interação com o usuário, controlando o navegador e alterando o conteúdo
do documento que aparece na janela do navegador.
Utilizando a interação com o navegador que o Javascript proporciona, é possível movimentar
objetos pelo navegador, adicionar conteúdo, criar novos objetos, fazer desenhos, guardar
informações, entre outras coisas, bastando manipular o conteúdo HTML da página.

O padrão HTML é uma linguagem de marcação de hipertexto que é utilizado como base para a internet
atual. O HTML mudou bastante desde sua versão inicial em 1991
\cite{powell2003html}: No início o HTML não era muito definido, e não tinha um padrão, até que que a
IETF começou a padronizá-lo em 1995, lançando a
primeira versão padronizada do HTML, que ficou conhecida como HTML 2.0.
Com o tempo o HTML foi se tornando mais padronizado, ganhando validações e novas integrações
(com CSS por exemplo), tornando-o mais versátil e adequado às novas necessidades da
web. Outra vantagem que a padronização trouxe foi a facilidade de implementação pelos
navegadores, que agora implementavam com menos frequência o seu próprio estilo de marcação.
A ultima versão do HTML desenvolvida é o HTML5, que traz funcionalidades que dão um
olhar diferente para a internet, ou seja, ao invés de apenas se preocupar com texto
e formatação, o HTML5 visa atender às necessidades multimídia da internet, cobrindo
todos os espaços multimídia que tiveram que ser contornados com outras tecnologias,
como por exemplo vídeo, audio e desenho.
Com o crescimento do HTML, surgiu a necessidade de separar o conteúdo
da visualização, e essa foi a principal motivação para o nascimento de
uma linguagem intermediária, e para isso o CSS foi criado.
Com essa criação foi possível remover algumas tags do HTML que eram utilizadas
para a formatação, como a tag font por exemplo, e esse tipo de
estilização começou a ser feita apenas com CSS.

CSS é uma linguagem para criação de folhas de estilo, que interage com um documento
HTML, dando a ele uma melhor visualização, manipulando fontes, cores e espaçamentos.
A versão 2 do CSS se tornou padrão em maio de 1998 \cite{zeldman2009designing},
e atualmente está sendo desenvolvida a versão 3, que já está parcialmente implementada
nos navegadores compatíveis com HTML5, apesar de ainda estar em construção (assim como o HTML5).

Utilizando essas tecnologias emergentes da web é possível fazer jogos casuais simples
e divertidos. Uma das definições que ajuda a resumir um jogo casual
mencionada por \citeonline{trefry2010casual} é:

\begin{singlespacing}
\begin{citacao}{4cm}{0cm}\footnotesize \emph
    ``Um jogo casual não exige do jogador horas de
      dedicação para cada seção de jogo, o jogo deve ser jogado aos poucos, em pequenas
      porções, isso vai ajudar o jogo a se adequar a quantidade de tempo que os jogadores
      de jogos casuais podem dedicar a ele.''
\end{citacao}
\end{singlespacing}

A tecnologia que domina o mercado de jogos casuais atualmente é o Adobe Flash, que devido
a sua grande adoção na época que a internet carecia de multimídia, se tornou "padrão"
em todos os computadores atuais, apesar de ser um software de terceiros instalado
como plugin para os navegadores. Java tem uma pequena porcentagem no total de jogos
na web, e é pouco usada ultimamente, diferentemente do framework Unity, que pode exportar
os games para a web, e tem crescido bastante desde seu lançamento.

O HTML5 traz a possibilidade de suprir várias funcionalidades de cada uma dessas tecnologias,
com a vantagem de ser um padrão implementado pelas empresas que desenvolvem os navegadores
mais conhecidos do mercado, e que já estão fazendo seus versões para celulares e
video games, abrindo mais possibilidades para um jogo simples estar disponível em mais plataformas.

O objetivo desse trabalho é apresentar os conceitos e utilizações das novas ferramentas utilizadas
nos navegadores compatíveis com HTML5, focando principalmente em HTML5
e Javascript, mostrando as possibilidades de inovação e negócio que as novas tecnologias implementadas
nos navegadores compatíveis com HTML5 trazem para o mundo de
desenvolvimento de jogos casuais, e comparando-as com as tecnologias existentes no mercado.

Será abordada a história do HTML, mostrando sua evolução com o passar do tempo, chegando
a seu estado atual, explorando quais as suas finalidades e quais os problemas que ele foi criado para resolver.
Logo em seguida será abordada a história do Javascript e sua evolução até seu estado
atual, mostrando qual a sua ligação com o HTML e como a sua evolução é importante
para o futuro da web, não somente com páginas comuns, mas com experimentos multimídia também.

Será apresentada também uma introdução ao CSS e seu uso, para assim formar "o ciclo
do HTML5", que é formado por HTML, Javascript e CSS.

Com uma visão maior de HTML5, o trabalho vai focar em seus recursos e como isso pode
ser usado para o desenvolvimento de games. O principal objetivo é
descrever cada um de seus recursos-chave e mostrar passo a passo, com exemplos, onde ele pode ser aplicado e porque isso
é viável e útil, comparando com as tecnologias existentes no mercado atualmente.

Para mostrar as características do HTML5 para o desenvolvimento de
games e sua aplicação o trabalho foi dividido em quatro partes
distintas. A primeira parte conta um pouco sobre o nascimento e
evolução das principais ferramentas para o desenvolvimento de jogos
nesse campo, a segunda mostra um pouco sobre a definição de jogos casuais e como
eles evoluiram desde seu nascimento. Após falar um pouco de história e
passar um pouco de definições, será mostrado na terceira parte as
principais ferramentas para se construir um jogo utilizando as
ferramentas que o HTML5 provê. Para finalizar, na quarta parte é
mostrada uma comparação das tecnologias que são utilizadas atualmente
com o HTML5, mostrando as vantagens e desvantagens de cada tecnologia,
utilizando itens como performance, dependências, portabilidade,
recursos e viabilidade para a comparação.

Com o grande crescimento da tecnologia dos navegadores, muita coisa se tornou possível
e viável, uma delas é o desenvolvimento de jogos casuais. Com a constante evolução
dos interpretadores de Javascript, HTML e CSS dos navegadores mais populares, é possível
desenvolver jogos simples e divertidos e conseguir um bom retorno com isso. Atualmente,
várias tecnologias são utilizadas para se fazer jogos para a web, cada uma com suas
vantagens e limitações, e o HTML5 (e seus complementos), entrou na competição, prometendo
se tornar o padrão para multimídia na web.
