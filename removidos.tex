\subsection{Vantagens do HTML5}
Após as comparações feitas com as tecnologias que são utilizadas para
o desenvolvimento de jogos para a web na data de escrita desse
trabalho é possível chegar a uma conclusão sobre quais são
as vantagens do HTML5 atualmente.

Uma das vantagens claras que atrapalham muito os desenvolvedores de
jogos atualmente é a portabilidade, pois os desenvolvedores muitas
vezes precisam traduzir seu código para um grande quantidade de
plataformas para poder atingir uma quantidade maior de jogadores. O
problema com esse modelo é a fragmentação que isso gera, pois o mesmo
código estará dividido entre várias plataformas, com várias linguagens
de programação e provavelmente várias equipes de trabalho para manter
o jogo funcionando e corrigir cada problema que apareça. Quando um
problema é detectado na parte padrão do jogo todas as equipes devem
ser informadas e todos devem fazer determinada alteração e publicar o
jogo novamente para a plataforma em que o jogo está executando.

Utilizando HTML5 esse problema é resolvido com facilidade, pois caso
um problema seja encontrado na parte padrão do jogo, ele vai ser
corrigido no jogo principal e enviado para o ambiente de produção, com
isso todos os jogadores terão o jogo atualizado assim que acessarem o
jogo em um navegador com conexão com a internet. Isso facilita muito a
distribuição dos jogos, apenas cabendo ao desenvolvedor tratar os
tamanhos de telas e dispositivos de entrada no seu jogo.

A grande maioria dos jogadores possuem acesso a internet e geralmente
estão conectados em suas redes sociais ou fazendo tarefas corriqueiras
tanto em casa como em lugares monótonos como uma fila de banco por
exemplo. Se o jogo foi desenvolvido com HTML5, o jogador muito
provavelmente vai ter esse jogo disponível onde ele estiver, sem
depender de uma determinada plataforma (como um computador de mesa por
exemplo) para que continue seu jogo ou pelo menos o distraia num
momento monótono que precisa ser enfrentado. Com o crescimento do uso
de celulares e tablets o uso de tecnologias que façam o produto final
ficar disponível para essas plataformas também se faz cada vez mais
necessária para conseguir um publico maior.

Uma outra vantagem é dar ao jogador a liberdade de utilizar o
navegador que mais lhe agrada para poder rodar o jogo desenvolvido,
deixando apenas informado para o jogador quais são as funcionalidades
necessárias para que o jogo seja executado, e algunas recomendações
sobre navegadores. Apesar do jogo provavelmente funcionar bem sobre o
navegador padrão do usuário (se ele estiver devidamente atualizado
para a ultima versão disponível), alguns jogadores podem não gostar de
utilizá-lo, nesse caso ele pode simplesmente utilizar
o seu navegador preferido, que se estiver em sua ultima
versão e possuir suporte as funcionalidades mais novas do HTML5 será
capaz de executar o jogo sem maiores problemas. Outra vantagem disso é
dar a oportunidade do jogador testar qual o navegador que possui uma
maior performance na plataforma que ele está utilizando, dando ao
jogador uma grande gama de opções para que ele possa escolher a opção
que mais lhe convém.

\subsection{Desvantagens do HTML5}
Assim como toda tecnologia o HTML5 também possui suas desvantagens
sobre seus concorrentes, e muitas vezes as vantagens se tornam
desvantagens dependendo do ponto de vista do desenvolvedor.

A vantagem de se ter apenas um código para todas as plataformas é
considerada desvantagem por alguns desenvolvedores, pois dessa maneira
muitas condições devem ser tratadas para que o jogo funcione bem em
todas as plataformas, e isso torna o código difícil de ser mantido
dependendo de como o desenvolvedor organizá-lo.
Esse mesmo problema acontece com desenvolvedores de aplicações para o
sistema operacional Android, que precisam tratar as diferentes
configurações de \textit{hardware} de cada dispositivo, suas
velocidades de processamento, tamanho de tela e etc. Alguns
desenvolvedores (como é o caso da Rovio, criadora do jogo Angry Birds)
decidiram fazer diversos jogos para diversos aparelhos para evitar
esse tipo de fragmentação.

Por ser uma tecnologia nova ainda não existem muitas ferramentas para
auxiliar o desenvolvedor no desenvolvimento de jogos, exigindo do
mesmo um maior contato com o código e a execução de algumas tarefas
manualmente. Um exemplo disso é a criação de sprites para um jogo 2D,
que precisa ser feita em um editor de imagens padrão e mapeada
manualmente para o código, para que seja possível saber em qual parte
da imagem está cada sprite do jogo.
Atualmente há algumas ferramentas que estão sendo criadas pelos
próprios desenvolvedores para facilitar essa tarefa, assim como as
outras que ainda fazem falta no HTML5, ou seja, é uma questão de tempo
para que essa tecnologia receba cada vez mais boas ferramentas para
facilitar esses trabalhos manuais.

Uma desvantagem que não pode ser controlada pelo desenvolvedor são as
atitudes do usuário com a atualização do navegador, sendo assim, caso
o usuário possua um navegador desatualizado o jogo provavelmente não
funcionará como deveria, e dependendo das funcionalidades que ele
necessita ele não chegará a ser executado. Para reduzir esse problema,
as empresas que desenvolvem os navegadores estão adotando uma forma
transparente de atualizações nos navegadores para que o usuário não
precise se preocupar em manter seu navegador atualizado todo o tempo,
deixando essa preocupação com as empresas que os desenvolvem.
A primeira empresa a utilizar o sistema de atualização automática do
navegador foi a Google, que automaticamente atualiza o Google Chrome
de seus usuários, assim deixando-os sempre com a ultima versão estável
do navegador. Essa configuração pode ser desabilitada para os casos em
que uma determinada versão precisa ser utilizada, mas para o caso
geral dos usuários domésticos essa é uma ótima funcionalidade, pois dá
ao desenvolvedor web uma maior gama de possibilidades para os usuários
desse navegador, podendo assumir que ele está utilizando a ultima
versão estável.
A Mozilla após ver o bom modelo de atualização do Google Chrome
decidiu também utilizá-lo no seu navegador, o Firefox. Os usuários
também começam a receber uma nova versão a cada quatro meses, e sempre
estarão utilizando as novas versões do navegador, além de fazer várias
campanhas para que as pessoas que ainda utilizam as versões antigas do navegador (que ainda
não possuía suporte a atualização automática) atualizem o mesmo para
se beneficiar dessa nova funcionalidade.
