\section{Jogos casuais e sua evolução}

Nesse capítulo será abordado o que são os jogos casuais, como eles
evoluíram com o passar dos tempos, qual o estado desse tipo de jogo
desenvolvido para internet e a influencia deles no ritmo de vida das
pessoas.

\subsection{Os jogos casuais}

É possível definir jogos casuais como aquele jogo que pode ser jogados em um curto período
de tempo. Conforme \citeonline{trefry2010casual}, são jogos rápidos e acessíveis.
Diferentemente dos grandes jogos criados por grandes e conceituadas
empresas do mundo dos jogos, os jogos casuais não precisam de grandes
equipes para serem desenvolvidos, uma equipe pequena de 3 ou 4 pessoas
pode fazer um jogo simples e divertido, e em alguns casos apenas um
desenvolvedor pode criar um jogo casual se possuir habilidades em
diferentes áreas (como design e programação).

Uma vantagem dos jogos casuais é a facilidade em divulgá-lo. De acordo com
\citeonline{ozcan2010recent}:

\begin{singlespacing}
\begin{citacao}{4cm}{0cm}\footnotesize \emph
    ``O tempo de entrada para os jogos casuais em navegadores é pequeno,
    pois eles são fáceis de se recomendar para os amigos. Pessoas jogando
    algum jogo de navegador são uma ótima maneira de atrair mais
    pessoas para jogar um determinado jogo.''
\end{citacao}
\end{singlespacing}

Atualmente, os jogos casuais possuem muitas vantagens, pois a conexão
com os amigos é feita muito facilmente, portanto, os jogadores
recomendam uns para os outros o jogo que está jogando. Um dos motivos
que falicilita essa proliferação é a portabilidade, pois o jogo pode
estar em um dispositivo móvel, ou em um computador qualquer, acessível
via internet, e etc.

\cite{ozcan2010recent} já comentava sobre a facilidade que a
portabilidade desse tipo de jogo trazia para a vida das pessoas, e
consequentemente para o desenvolvedor. Segundo ele a portabilidade é um
dos grandes motivos do sucesso desses jogos, pois a possibilidade de
jogar em qualquer lugar, em curtos períodos no meio da vida cotidiana
do jogador, faz total diferença para quem não é um jogador nato.

Tendo isso em mente, é possível imaginar, um jogo divertido e desafiador
sendo passado entre amigos, utilizando diversos meios, preenchendo os
tempos vagos de cada um, trazendo pequenos desafios divertidos que
trazem ao jogador a vontade de compartilhar e passar esse desafio para
o amigo utilizando o jogo, o que faz ele ser cada vez mais popular e
divertido.

\subsection{A evolução dos jogos casuais no navegador}
%TODO: procurar mais sobre a história dos jogos casuais

O numero de jogos de navegador cresceu nos últimos anos e esse
desenvolvimento continua \cite{ozcan2010recent}. Esse tipo de jogo
ganhou força com a ajuda de plugins externos como o Flash player e
os applets Java, que conseguiam um poder grande de processamento para
recursos da internet da época, e assim os jogos foram crescendo rapidamente.

Conforme \citeonline{trefry2010casual}, o termo "jogo casual" tem se
modificado bastante ao longo do tempo, e está sendo utilizado para
diversos tipos de jogos. Algumas características comuns entre os jogos
que são considerados casuais são:

\begin{itemize}
    \item Regras e objetivos devem ser claros;
    \item Os jogadores devem conseguir proeficiencia rapidamente;
    \item O jogo se adapta a vida e a agenda do jogador;
    \item O conceito do jogo utiliza conteúdos familiares e coisas da vida.
\end{itemize}

O HTML5 tem crescido muito ultimamente para fazer muitas das coisas
que só seria possível fazer com plugins externos como o Flash, e aos
poucos está mostrando que é possível desenvolver jogos casuais sem
precisar de um plugin externo para isso.
