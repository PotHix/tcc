\section{Jogos casuais e sua evolução}

Nesse capítulo será abordado o que são os jogos casuais, como eles
evoluíram com o passar dos tempos, qual o estado desse tipo de jogo
desenvolvido para internet e a influencia deles no ritmo de vida das
pessoas.

\subsection{Os jogos casuais}

É possível definir jogos casuais como aquele jogo que pode ser jogado em um curto período
de tempo. Conforme \citeonline{trefry2010casual}, são jogos rápidos e acessíveis.
Diferentemente dos grandes jogos criados por grandes e conceituadas
empresas do mundo dos jogos, os jogos casuais não precisam de grandes
equipes para serem desenvolvidos: uma equipe pequena de 3 ou 4 pessoas
pode fazer um jogo simples e divertido, e em alguns casos apenas um
desenvolvedor pode criar um jogo casual se possuir habilidades em
diferentes áreas (como design e programação).

Uma vantagem dos jogos casuais é a facilidade em divulgá-lo. De acordo com
\citeonline{ozcan2010recent}:

\begin{singlespacing}
\begin{citacao}{4cm}{0cm}\footnotesize \emph
    ``O tempo de entrada do jogador nos jogos casuais em navegadores é pequeno,
    pois eles são fáceis de se recomendar para os amigos. Pessoas jogando
    algum jogo de navegador são uma ótima maneira de atrair mais
    pessoas para jogar um determinado jogo.''
\end{citacao}
\end{singlespacing}

Atualmente, os jogos casuais possuem muitas vantagens, pois a conexão
com os amigos é feita muito facilmente, portanto os jogadores
recomendam uns para os outros o jogo que está jogando. Um dos motivos
que facilita essa proliferação é a portabilidade, pois o jogo pode
estar em um dispositivo móvel, ou em um computador qualquer, acessível
via internet, e etc.

\citeonline{ozcan2010recent} já comentava sobre a facilidade que a
portabilidade desse tipo de jogo trazia para a vida das pessoas, e
consequentemente para o desenvolvedor. Segundo ele, a portabilidade é um
dos grandes motivos do sucesso desses jogos, pois a possibilidade de
jogar em qualquer lugar, em curtos períodos no meio da vida cotidiana
do jogador, faz total diferença para quem não é um jogador nato.

Tendo isso em mente, é possível imaginar um jogo divertido e desafiador
sendo passado entre amigos, utilizando diversos meios, preenchendo os
tempos vagos de cada um, trazendo pequenos desafios divertidos que
trazem ao jogador a vontade de compartilhar e passar esse desafio para
o amigo utilizando o jogo, o que faz ele ser cada vez mais popular e
divertido.

\subsection{A evolução dos jogos casuais no navegador}

Podemos dizer que os jogos casuais para computador começaram com
o paciência criado para o Windows pela Microsoft em 1990 \cite{trefry2010casual}.
Nessa época o mouse estava começando a ser introduzido no mercado e
eles precisavam de algo que fizesse as pessoas entenderem como o mouse
funcionava e quais suas utilidades. O paciência fez um ótimo trabalho
nessa área e inclusive hoje ele pode se considerar o jogo mais jogado
do mundo conforme \citeonline{trefry2010casual}.

Conforme \citeonline{trefry2010casual}, o termo "jogo casual" tem se
modificado bastante ao longo do tempo, e está sendo utilizado para
diversos tipos de jogos. Algumas características comuns entre os jogos
que são considerados casuais são:

\begin{itemize}
    \item Regras e objetivos devem ser claros;
    \item Os jogadores devem conseguir proficiência rapidamente;
    \item O jogo se adapta à vida e à agenda do jogador;
    \item O conceito do jogo utiliza conteúdos familiares e coisas da vida.
\end{itemize}

Várias grandes empresas têm investido em jogos casuais. Um grande
exemplo é a Nintendo com o Nintendo Wii, que grande parte dos jogos
são casuais, simples de aprender a jogar e divertidos para se jogar
em um curto período de tempo.
A Microsoft tem feito o mesmo com o Kinect, que
possui como jogo principal (o jogo que apresentou a plataforma) um
conjunto de jogos casuais que são simples e divertidos para se jogar
em grupos.

Quando os jogos começaram a ser desenvolvidos para executar
diretamente no navegador, era fácil de ver que a grande maioria deles
se enquadravam na descrição de jogos casuais que é usada nesse
trabalho, e ainda hoje, a grande maioria desses jogos segue
esse mesmo padrão, e há uma explicação simples para esse
acontecimento: quando um desenvolvedor pensa em fazer um jogo que
não seja casual, ou seja, um jogo que exija mais tempo, dedicação ou
aprendizado do jogador, geralmente ele não pensa em fazer um jogo para
navegador porque o público-alvo que está disposto a jogá-lo muito
provavelmente estará em frente a um computador ou console para
fazê-lo.
O publico alvo que joga no navegador é um publico que está em uma hora
vaga, possivelmente no horário de almoço no trabalho, na fila de um
banco pelo celular, aguardando alguém para sair e etc.

O numero de jogos de navegador cresceu nos últimos anos e esse
desenvolvimento continua \cite{ozcan2010recent}. Esse tipo de jogo
ganhou força com a ajuda de plugins externos como o Flash player e
os applets Java, que utilizando uma extensão instalada no computador
do usuário para executar código que chega pelo navegador,
conseguiam um poder grande de processamento comparado aos
recursos da internet da época.
Atualmente a situação continua a mesma, apesar dos applets Java não
serem tão populares, mas o Flash ainda domina o mundo dos jogos
casuais até o momento, devido a sua performance e alta adoção pelos
usuários e desenvolvedores.

O HTML5 tem crescido muito, ganhando várias novas funcionalidades e
implementações pelos principais navegadores, e atualmente já é possível utilizá-lo
para fazer muitas das coisas que só seria possível fazer com plugins
externos como o Flash. Aos poucos ele está mostrando que é possível
desenvolver jogos casuais sem precisar de um plugin externo para isso.
Alguns jogos foram desenvolvidos utilizando essa tecnologia, cada um
utilizando o melhor de algumas funcionalidades e mostrando que é
totalmente plausível a sua utilização. Nos próximos capítulos serão
apresentados jogos como exemplo das funcionalidades do HTML5.
