\section{Jogos casuais e sua evolução}

Nesse capítulo será abordado o que são os jogos casuais, como eles
evoluíram com o passar dos tempos, qual o estado desse tipo de jogo
desenvolvido para internet e a influencia deles no ritmo de vida das
pessoas.

\subsection{Os jogos casuais}

É possível definir jogos casuais como aquele jogo que pode ser jogados em um curto período
de tempo, conforme \citeonline{trefry2010casual}, são jogos rápidos e acessíveis.
Diferentemente dos grandes jogos criados por grandes e conceituadas
empresas do mundo dos games, os games casuais não precisam de grandes
equipes para serem desenvolvidos, uma equipe pequena de 3 ou 4 pessoas
pode fazer um jogo simples e divertido, e em alguns casos apenas um
desenvolvedor pode criar um jogo casual se possuir habilidades em
diferentes áreas (como design e programação).

Uma vantagem dos jogos casuais é a facilidade em divulgá-lo. De acordo com
\citeonline{ozcan2010recent}:

\begin{singlespacing}
\begin{citacao}{4cm}{0cm}\footnotesize \emph
    ``O tempo de entrada para os jogos casuais em navegadores é pequeno,
    pois eles são fáceis de se recomendar para os amigos. Pessoas jogando
    algum jogo de navegador são uma ótima maneira de atrair mais mais
    pessoas para jogar um determinado jogo.''
\end{citacao}
\end{singlespacing}

Atualmente, os jogos casuais possuem muitas vantagens, pois a conexão
com os amigos é feita muito facilmente, portanto, os jogadores
recomendam uns para os outros o jogo que está jogando.
Um dos grandes motivos do sucesso desses jogos é a possibilidade de
jogar em qualquer lugar, em curtos períodos no meio da vida cotidiana
do jogador, onde alguns movimentos no jogo vão tomar apenas alguns
minutos. \cite{ozcan2010recent}.
Tendo isso em mente, é possível imaginar, um jogo divertido e desafiador
sendo passado entre amigos, preenchendo os tempos vagos de cada um,
pois é acessível em qualquer lugar, e o jogador é constantemente
desafiado pelo amigo utilizando o jogo, e isso alimenta cada vez mais
o jogo e influencia no divertimento.

\subsection{A evolução dos jogos casuais no navegador}
%TODO: procurar mais sobre a história dos jogos casuais

O numero de jogos de navegador cresceu nos últimos anos e esse
desenvolvimento continua \cite{ozcan2010recent}. Esse tipo de jogo
ganhou força com a ajuda de plugins externos como o Flash player e
os applets Java, que conseguiam um poder grande de processamento para
recursos da internet da época, e assim os jogos foram crescendo rapidamente.

O HTML5 tem crescido muito ultimamente para fazer muitas das coisas
que só seria possível fazer com plugins externos como o Flash, e aos
poucos está mostrando que é possível desenvolver jogos casuais sem
precisar de um plugin externo para isso.
