\section{Jogos casuais e sua evolução}
% Utilizar esse capítulo para falar bastante sobre jogos no navegador
% por que aqui começa o trabalho de verdade

\subsection{Os jogos casuais}
Um dos grandes motivos do sucesso desses jogos é a possibilidade de
jogar em qualquer lugar, em curtos períodos no meio da vida cotidiana
do jogador, onde alguns movimentos no jogo vão tomar apenas alguns
minutos \cite{ozcan2010recent}.

\subsection{A evolução dos jogos casuais no navegador}
O numero de jogos de navegador cresceu nos últimos anos e esse
desenvolvimento continua \cite{ozcan2010recent}. Esse tipo de jogo
ganhou força com a ajuda de plugins externos como o Flash player ou
o Java applets, que conseguiam um poder grande de processamento para
recursos da internet da época, e assim os jogos foram crescendo rapidamente.

Os jogos desenvolvidos para a internet geralmente são jogos simples,
que podem prender o jogador por curtos períodos de tempo, permitindo
ao mesmo jogar nos intervalos da vida cotidiana, como o horário de almoço
do trabalho, por exemplo.
Por possuir fácil acessibilidade e estar na internet juntamente com as
outras páginas e redes sociais, os jogos possuem uma fácil integração
com redes sociais, assim ganhando facilmente a atenção desses
jogadores casuais, que ao estar navegando em sua rede social, recebem
mensagens de seus amigos sobre um jogo que o mesmo está jogando,
desafiando-o para vencer o seu recorde, ou ajudá-lo em alguma missão
do jogo e etc.

O HTML5 tem crescido muito ultimamente para fazer muitas das coisas
que só seria possível fazer com plugins externos como o Flash, e aos
poucos está mostrando que é possível desenvolver jogos casuais sem
precisar de um plugin externo para isso.
