\section{Javascript, HTML5 e a evolução dos navegadores}

%TODO: Fazer um resumo da arquitetura da internet

Os navegadores evoluíram muito nos últimos anos e com essa evolução
veio a possibilidade de utilização de tecnologias que integram
diretamente com o navegador, provendo muitas possibilidades de uso com
a vantagem de terceirizar o trabalho da implementação multi-plataforma
de uma aplicação, pois as empresas que desenvolvem os navegadores
terão todo esse trabalho pelo desenvolvedor. Essa se tornou uma das
grandes vantagens do HTML5, e com sua evolução constante, é possível
construir diretamente no navegador muitas coisas que atualmente necessitam
de ferramentas de terceiros, apenas utilizando javascript para a
manipulação.

\subsection{HTML e sua evolução}

HTML é uma linguagem de marcação de hipertexto que é utilizada como base para a internet
atual. O HTML mudou bastante desde sua versão inicial em 1991 \cite{powell2003html}.
No início o HTML não era muito definido, e cada
navegador definia novas tags que tinham funções diferentes,
não seguindo um padrão específico, o que dificultava
bastante o trabalho dos desenvolvedores da época.
O tempo se foi se passando e após algum tempo um esforço de padronização da linguagem
começou a ser organizado e, em 1995 o IETF começou o processo de
padronização, lançando a primeira versão padronizada do HTML, que ficou conhecida como HTML 2.0.
Com o tempo o HTML foi se tornando mais padronizado, ganhando validações e novas integrações
(com CSS por exemplo), tornando-o mais versátil e adequado às novas necessidades da
internet.
Essa padronização resolveu vários problemas de fragmentação, onde cada
navegador implementava sua funcionalidade específica, pois os agora
eles poderiam começar a seguir as especificações padronizadas,
facilitando as implementações de cada navegador, que agora implementavam com
menos frequência o seu próprio estilo de marcação.
As primeiras página HTML eram feitas apenas com HTML, ou seja, era
responsabilidade do HTML cuidar de toda a diagramação do site para que
ele fosse agradável para o visitante, e muitos tipos de
desenvolvimento foram adotados, utilizando os melhores elementos para
conseguir diagramar uma página sem se importar com a semântica do
documento. Com a chegada do site de busca da Google as coisas mudaram bastante, pois
agora não havia apenas humanos olhando para a página, robôs também
estavam passando por ali e precisavam entender da melhor maneira
possível aquele conteúdo para poder adicionar a página no lugar certo
nos índices de busca. Com isso o HTML começa a ser utilizado de uma
forma mais estruturada, utilizando o CSS para estilizar a visualização,
assim, removendo do HTML a responsabilidade de lidar com cores, tipos de
fonte e etc. Com as responsabilidades bem divididas, o CSS pode tratar
do que é mostrado para o usuário e o HTML cuidar do que é mostrado
para os robôs, mantendo o melhor dos dois pontos.

%TODO: falar de XHTML
Um pouco depois da criação do HTML foi visto uma necessidade de
criação de um documento que tivesse sua estrutura mais estrita,
ou seja, que tivesse mais restrições que o atual HTML para reduzir a
fragmentação causada pela adição de novos elementos criados geralmente
pela Microsoft e Netstape. Visando essa necessidade, foi criado o XHTML,
que é derivado do XML e herda várias características do mesmo,
inclusive o tipo de validação do documento.
O XHTML fez sucesso durante um bom tempo devido a sua sintaxe concisa e
suas validações consistentes, e ainda hoje é usado, apesar de estar
perdendo campos para a nova versão do HTML, o HTML5, enquanto é
desenovlvida a nova versão do XHTML.

A ultima versão do HTML que está em desenvolvimento é o HTML5, que
traz novos marcadores para definir melhor a semântica dos documentos e
traz também funcionalidades que dão um olhar diferente para a internet,
ou seja, ao invés de apenas se preocupar com texto e formatação, o HTML5
visa atender às necessidades multimídia da internet, cobrindo todos os
espaços multimídia que tiveram que ser contornados com outras tecnologias,
como por exemplo vídeo, áudio e desenho.

Ainda é difícil dizer o que é parte do HTML5, pois atualmente o HTML5
é apenas um rótulo utilizado para identificar um conjunto de recursos
obtidos através do navegador, entre esses recursos está a nova
linguagem de marcação, que possui diversas novas funcionalidades para
deixar um documento mais semântico.
Alguns exemplos das novas funcionalidades que compõe esse novo ambiente:

\begin{itemize}
  \item Nova tag <article> para delimitar artigos;
  \item Nova tag <section> para delimitar seções;
  \item Nova tag <aside> para delimitar menu lateral;
  \item Nova tag <audio> para inserir áudio na pagina;
  \item Nova tag <video> para inserir vídeo na pagina;
  \item Novos tipos para a tag <input> para inserir mais facilmente:
  data, url, numero, dia, mês, ano, horário, email e outros;
  \item History API, para poder ter acesso ao histórico de visitas do
  usuário, pedindo-lhe permissão para esse acesso;
  \item GeoLocation: Utilizado para saber onde o usuário está localizado no
  mundo;
  \item WebWorkers: Utilizado para executar scripts independentemente das ações
  da página;
  \item WebSockets: Utilizado para criar uma conexão persistente entre
  o servidor e o navegador;
  \item LocalStorage: Utilizado para guardar informações no navegador
  do usuário;
  \item Offline cache: Utilizado para salvar os arquivos necessários
  para que a aplicação fique disponível quando o usuário não estiver
  conectado à internet;
  \item Canvas: Para fazer desenhos e importar imagens de uma maneira
  mais fácil, e rápida, não dependendo da separação de objetos do DOM;
  \item SVG: Para criação e manipulação de vetores diretamente no
  navegador;
\end{itemize}

\subsection{Javascript e sua evolução}

Javascript é uma linguagem que foi criada em 1995 por Brendan Eich,
um engenheiro da Netscape, e foi lançada no começo de 1996 com o
Netscape 2 \cite{mdnjavascript}, sendo padronizada pela Ecma
Internacional em 1997, lançando assim a primeira versão do ECMAScript.
Javascript é uma linguagem de programação com capacidades de orientação a objetos e que tem como uma das suas principais
utilizações a manipulação de objetos em navegadores \cite{flanagan2006javascript}, sendo que, nesse contexto,
ela tem seu propósito estendido com objetos que permitem aos scripts uma interação com o usuário,
controlando o navegador e alterando o conteúdo do documento que aparece na janela do navegador.
Desde sua padronização, o Javascript vem sendo alterado e
recebeu algumas atualizações, sendo que a versão comumente
utilizada (e considerada estável) é a versão 3, padronizada em 1999.
O Javascript é muito utilizado nas paginas de internet atuais para
diversas finalidades, seja validação de simples formulários até
animações mais complexas. Várias bibliotecas foram desenvolvidas para
dar um poder maior ao desenvolvedor, sendo uma das mais conhecidas o
jQuery, que ajuda o desenvolvedor a ser mais produtivo, facilitando a
pesquisa por objetos dentro do DOM, facilitando a criação de efeitos
básicos, dando uma interface simples para alterar CSS e outras coisas.

\subsection{CSS e sua evolução}

CSS é uma linguagem para criação de folhas de estilo, que foi criada
para facilitar a criação de páginas para internet, comparada as técnicas da época de
1990, o CSS fornece um controle bem maior sobre as páginas \cite{schmitt2009css}.
O CSS interage com um documento HTML, dando a ele uma melhor visualização, manipulando
fontes, cores e espaçamentos, assim deixando a diagramação e visualização do documento
separado da estrutura do mesmo, deixando para o HTML apenas a
responsabilidade de manter um documento estruturado e semântico.
A versão mais utilizada atualmente, e considerada estável para utilização, é a versão 2
que se tornou padrão em maio de 1998 \cite{zeldman2009designing}.
A versão 3 está em desenvolvimento e alguns dos navegadores atuais mais populares
já suportam muitas de suas funcionalidades, e estão constantemente
adicionando as novas funcionalidades assim que são lançadas.
Com CSS3 é possível fazer muitas coisas que demandavam muito trabalho
do desenvolvedor com CSS2. Um exemplo bem simples do quanto o CSS3
pode ajudar na criação de um layout são os botões com cantos
arredondados, que era muito trabalhoso de se fazer utilizando CSS2 e
dependia de várias imagens juntas para fazer cada um dos cantos e
bordas do botão. O CSS3 possui essa funcionalidade implementada, e os
botões podem ter cantos arredondados facilmente utilizando a
propriedade \textit{border-radius}, tirando toda essa complexidade das mãos do
desenvolvedor.

\subsection{A evolução dos navegadores}

Os navegadores nasceram muito antes da interface gráfica,
e é possível notar sua constante evolução para suprir as necessidades
dos usuários e desenvolvedores \cite{robbins2006web}. No início, os
navegadores eram apenas texto preto sobre um fundo cinza, e foi assim
nos primórdios da internet, enquanto o Mosaic e o Lynx eram os
navegadores modo texto famosos, sendo que o Mosaic foi o primeiro
navegador a mostrar as imagens junto com o texto.
Em 1994 foi lançado o Netscape 0.9, que
trouxe várias inovações para a internet, e dois anos depois a
Microsoft contra ataca com o Internet Explorer 3.0, com várias
funcionalidades novas, inclusive o suporte a folhas de estilo (e
introduzindo o CSS), e iniciando o que chamam de "A guerra dos
navegadores". Essa guerra foi travada entre a Microsoft e a Netscape
por muitos anos e trouxe muitas coisas boas e ruins para a internet. O
impacto positivo foi a aceleração do desenvolvimento para essas
plataforma, e o impacto negativo foi a grande falta de padrão para os
desenvolvedores, pois cada navegador implementava HTML utilizando seu
próprio formato. A Microsoft conseguiu uma grande vantagem quando lançou
o Windows 95 OSR2 que já tinha o internet explorer gratuitamente
\cite{asleson2006foundations}, e venceu oficialmente a guerra dos navegadores quando a
Netscape foi comprada pela AOL(America Online). A Netscape deixou o código-fonte
de seu navegador liberado para quem quisesse continuar o projeto,
e a Mozilla utilizou esse código para começar o desenvolvimento
do que futuramente seria o Firefox.
Em 2005 foi lançada a primeira versão do Firefox e a Microsoft começa
a perder seu reinado absoluto com navegadores. O Firefox conseguiu uma
boa fatia do mercado de navegadores, e a Microsoft ficou para trás em
termos de tecnologia e segurança a partir da versão 6 do Internet Explorer.

Com o passar dos anos, outros navegadores entram nessa disputa, entre
eles estão o Opera, que tem uma pequena parte do mercado mas está em
constante desenvolvimento, e o Chrome da Google, que está entre os
mais rápidos e seguros navegadores da atualidade.
Novamente uma Guerra dos navegadores começa, mas dessa vez um pouco
mais controlada, pois existe um órgão regulamentador da internet que
todos os navegadores procuram seguir e se adequar, ele se chama W3C,
e seu papel nessa história é dizer o que
é padrão e o que não é, assim fazendo que a interface para os
desenvolvedores seja o mais parecida possível entre todos os
navegadores.
Ainda hoje cada navegador possui um conjunto diferente de funcionalidades
do HTML5 ou CSS3, o que faz com que nem todos os recursos funcionem em todos
os browsers, e sua própria implementação do
interpretador de Javascript, o que faz alguns serem mais rápidos que
outros. E uma das grandes vantagens atuais, principalmente falando de
aplicações ricas, que tem um uso intenso de Javascript, é a velocidade
desse interpretador e o suporte às novas funcionalidades que o HTML5
propõe.
Os principais navegadores da atualidade são: Opera, Firefox, Internet
Explorer e Chrome. Todos estão implementando, cada um em seu ritmo, as
funcionalidades do HTML5, e estes, em suas ultimas versões, serão usados
como base para os estudos desse trabalho.
