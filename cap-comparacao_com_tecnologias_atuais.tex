\section{Comparação com as tecnologias atuais}

Com base nas funcionalidades apresentadas sobre as tecnologias que
fazem parte da especificação do HTML5, será feita uma comparação com
as principais ferramentas utilizadas para o desenvolvimento de jogos
na data da escrita desse trabalho.

\subsection{Tecnologias atuais}

As ferramentas atuais para desenvolvimento de jogos para a internet se
utilizam de extensões que são instaladas nativamente no computador
para que possa ser uma plataforma na qual o jogo desenvolvido vai
executar sobre, para assim gerar uma plataforma mais consistente para
o desenvolvimento e ter um melhor acesso ao hardware do dispositivo em
questão. Essas tecnologias são:

\subsubsection{Java}

Java foi uma das primeiras tecnologias a ser aplicadas sobre os
navegadores, logo após a criação do Java, quando a tecnologia foi
portada uma facil execução sobre o navegador utilizando os chamados
applets.
%TODO: Escrever mais sobre os applets e sua utilização

\subsubsection{Flash}

Flash é um produto da Adobe para enriquecer a o visual de uma página
de internet, trazendo conteúdo multimídia. Atualmente, o Flash é a
tecnologia mais utilizada para desenvolvimento de jogos para a
internet, tendo sua extensão instalada em mais de 90 por cento dos
computadores pessoais.

%FIXME: Achar algumas referencias do flash como produto mais utilizado
%FIXME: Achar algumas referencias do flash com seu plugin instalado em mais de 90% dos navegadores
%TODO: Escrever mais sobre o flash

\subsubsection{Unity}

É uma tecnologia nova que está ganhando muito espaço no mercado de
desenvolvimento de jogos. Essa engine exporta o jogo desenvolvido para
diversas plataformas, entre elas o navegador, que utiliza um plugin um
plugin para executa-lo, assim como as tecnologias anteriores.
Uma das grandes vantagens dessa tecnologia é a facilidade para se
fazer jogos, pois esse é o foco da engine.

%TODO: Pesquisar mais sobre a Unity para dar uma visão mais ampla

\subsection{Performance}

No mundo dos jogos digitais, performance é um assunto muito falado,
pois os jogos consomem muito recurso do dispositivo em que estão
executando, pois estão, entre outras coisas, em constante execução, fazendo
calculos diversos para situar o jogador dentro do jogo, por exemplo.
Entre esses calculos, dependendo do jogo, estão calculos de fisica dos
objetos da cena e do próprio jogador, transformação de imagens para
dar impressão de movimento ou sombra e outras coisas que exigem alto
processamento em um curto período de tempo.
As tecnologias Java, Flash e Unity possuem uma performance melhor no
momento de escrita desse trabalho, pois eles possuem seu próprio
software desenvolvido, e roda nativamente no sistema operacional,
assim ganhando várias vantagens como um processo separado no sistema,
apenas executando um jogo que foi préviamente "compilado"
especialmente para essa tecnologia.

O HTML5 juntamente com Javascript perde um pouco em performance, pois
o sistema operacional vê apenas um navegador como processo, e esse
navegador vai fazer a interface entre o jogo e o sistema operacional.
Temos que levar em consideração que o navegador também tem seus
"processos" (outras páginas de internet que podem estar abertas no
momento), e ele deve gerenciá-las da melhor maneira.

%TODO: Continuar escrevendo essa parte, ainda está pela metade

\subsection{Portabilidade}

Sempre que um jogo casual é desenvolvido, tem-se em mente que ele deve
abranger a maior quantidade de dispositivos possível, para assim
atingir um maior publico, e ser utilizado onde o jogador estiver (como
é o caso dos dispositivos móveis). Quando a tecnologia para o
desenvolvimento do jogo é escolhida esse quesito tem que ser
ponderado, pois ele vai ditar em qual tipo de plataforma o jogo em
questão será executado, pois dependendo da plataforma escolhida o jogo ficará preso
a um determinado fabricante ou dispositivo.

%TODO: falar sobre as desvantagens de plugins e as vantagens dos navegadores

\subsection{Fragmentação}

Quando um jogo é desenvolvido para rodar em diversas plataformas
(conhecidos como jogos multiplataforma) e muito provável que o
desenvolvedor terá que lidar com um problema de fragmentação, pois
para cada tipo de dispositivo o jogo terá que ser customizado de uma
maneira diferente dependendo da tecnologia que foi escolhida para o
desenvolvimento do mesmo.
Utilizando HTML5 temos a vantagem de lidar com um navegador que estará
cuidando de todo o processo de aplicação nativa para o dispositivo,
funcionando como uma maquina virtual para executar o jogo
desenvolvido.

%TODO: falar sobre as outras tecnologias

\subsection{Recursos para desenvolvimento}

Cada tecnologia disponibiliza diferentes ferramentas para facilitar o
desenvolvimento em sua plataforma, para isso algumas utilizam meios
gráficos, como ferramentas WYSIWYG que facilitam o posicionamento de
objetos e outras apenas disponibilizam suas bibliotecas como um
framework para facilitar tarefas de código.

A adobe possui o Adobe Flash professional
professional(http://www.adobe.com/br/products/flash.html), que
possibilita ao desenvolvedor um acesso rapido as ferramentas de cores
e desenhos, e facilita muito o posicionamento e a transição entre o
código e a amostragem do jogo, integrando os dois lados em uma unica
plataforma, o que facilita o trabalho para o desenvolvedor
inesperiente.

O Unity editor (http://unity3d.com/unity/editor/) é o editor da engine Unity que possibilita toda a
criação do jogo utilizando a plataforma, permitindo, por exemplo, a
qualquer momento clicar no botão de iniciar e testar como está a sua
criação, emulando a plataforma que você deseja publicar.
Esse editor possui a facilidade de arrastar objetos e programá-los
facilmente, o que é totalmente necessário para um jogo 3D (que e o
foco do Unity).

Java não possui uma ferramenta diferente para o desenvolvimento de
jogos, apenas segue o mesmo padrão de desenvolvimento convencional
java e utiliza as bibliotecas de applet para preparar o jogo para
funcionar na web, ou seja, com uma das conhecidas IDEs (Eclipse por
exmeplo) você consegue desenvolver um jogo casual em Java.
%http://javaboutique.internet.com/tutorials/Java_Game_Programming/
