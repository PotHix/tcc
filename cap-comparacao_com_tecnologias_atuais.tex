\section{Comparação com as tecnologias atuais}

Com base nas funcionalidades apresentadas sobre as tecnologias que
fazem parte da especificação do HTML5, será feita uma comparação com
as principais ferramentas utilizadas para o desenvolvimento de jogos
na data da escrita desse trabalho.

\subsection{Tecnologias atuais}

As ferramentas atuais para desenvolvimento de jogos para a internet se
utilizam de extensões que são instaladas nativamente no computador
para que possa ser uma plataforma na qual o jogo desenvolvido vai
executar sobre, para assim gerar uma plataforma mais consistente para
o desenvolvimento e ter um melhor acesso ao hardware do dispositivo em
questão. Essas tecnologias são:

\subsubsection{Java}

Java foi uma das primeiras tecnologias a ser aplicadas sobre os
navegadores, logo após a criação do Java, quando a tecnologia foi
portada uma fácil execução sobre o navegador utilizando os chamados
applets.
%http://javaboutique.internet.com/tutorials/Java_Game_Programming/
%http://java.sun.com/applets/
%TODO: Escrever mais sobre os applets e sua utilização

\subsubsection{Flash}

Flash é um produto criado pela Macromedia em 1996, com a finalidade de
enriquecer o conteúdo da internet, trazendo funcionalidades que não
era possível fazer apenas com HTML na época de sua criação.
Com Flash, foi possível a criação de diversos sites com animações, e
foi o que originou as home pages animadas com um "Clique para entrar".

Em 2005 a Macromedia foi comprada pela adobe e vem sendo mantido por
eles desde então, juntamente com todo o pacote de multimidia da
Macromedia.

A facilidade em gerar animações com o Flash trouxe a idéia de
utilizá-lo para a criação de jogos que funcionassem diretamente no
navegador, tendo em vista que Flash o plugin exporta um binário para ser
executado utilizando o Flash player instalado pelo usuário, a
performance pode ser bem melhor, pois o flash player tem acesso ao
hardware do dispositivo no qual ele está instalado.

Atualmente, o Flash é a tecnologia mais utilizada para desenvolvimento de jogos para a
internet \cite{adobeflashleading}. Essa ferramenta possui muitas
bibliotecas feitas para facilitar o desenvolvimento de jogos, e por
possuir uma interface simples tanto para o designer como para o desenvolvedor,
ela vem mantendo seu legado ao longo dos anos.
Mais de 99 por cento dos computadores pessoais possuem o Flash Player
instalado \cite{adobeflashpenetration}. E esse é um dos grandes
motivos do sucesso do flash, pois quebra um dos grandes problemas de
se desenvolver um jogo que necessita de dependencias externas para ser
executado.

\subsubsection{Unity}

É uma tecnologia nova que está ganhando muito espaço no mercado de
desenvolvimento de jogos. Essa engine exporta o jogo desenvolvido para
diversas plataformas, entre elas o navegador, que utiliza um plugin um
plugin para executa-lo, assim como as tecnologias anteriores.
Uma das grandes vantagens dessa tecnologia é a facilidade para se
fazer jogos, pois esse é o foco da engine.

%TODO: Pesquisar mais sobre a Unity para dar uma visão mais ampla

\subsection{Performance}

Performance é um assunto muito discutido no desenvolvimento de jogos
pois os jogos consomem muito recurso do dispositivo em que estão
executando, pois estão, entre outras coisas, em constante execução, fazendo
cálculos diversos para situar o jogador dentro do jogo, por exemplo.
Entre esses cálculos, dependendo do jogo, estão os cálculos de física dos
objetos da cena e do próprio jogador, transformação de imagens para
dar impressão de movimento ou sombra e outras coisas que exigem alto
processamento em um curto período de tempo.

As tecnologias que exigem a instalação de um software de terceiros
(Java, Unity e Flash) possuem uma performance melhor no
momento de escrita desse trabalho. O motivo desse ganho de performance
é a otimização da plataforma para utilização em um determinado
dispositivo, fazendo com que o jogo seja interpretado e executado
nativamente no sistema operacional, assim ganhando várias
vantagens como um processo separado no sistema, apenas executando um
jogo que foi previamente "compilado" especialmente para essa tecnologia.

O HTML5 juntamente com Javascript perde um pouco em performance, pois
o sistema operacional vê apenas um navegador como processo, e esse
navegador vai fazer a interface entre o jogo e o sistema operacional.
O navegador não é uma plataforma otimizada para jogos (como é o caso
dos plugins, que possuem apenas essa finalidade, ou são bem
direcionados para o tal), apesar de estar e constante evolução e
desenvolvimento para interpretar Javascript de uma maneira mais veloz,
a cada nova versão.

O estado dos navegadores na data de escrita desse
trabalho suporta a execução de jogos jogos casuais simples, que exigem
um processamento aceitável do dispositivo, ou seja, jogos em duas
dimensões, com cálculos de física básica. Se o desenvolvimento do jogo
focar em performance, há uma grande possibilidade de que o mesmo
funcione sem grandes problemas e com uma performance aceitável em dispositivos
que possuem menor poder de processamento, como os celulares por exemplo.

\subsection{Portabilidade}

Sempre que um jogo casual é desenvolvido, tem-se em mente que ele deve
abranger a maior quantidade de dispositivos possível, para assim
atingir um maior publico, e ser utilizado onde o jogador estiver (como
é o caso dos dispositivos móveis). Quando a tecnologia para o
desenvolvimento do jogo é escolhida esse quesito tem que ser
ponderado, pois ele vai ditar em qual tipo de plataforma o jogo em
questão será executado, pois dependendo da plataforma escolhida o jogo ficará preso
a um determinado fabricante ou dispositivo.

No caso de jogos desenvolvidos para a web, esse problema acontece com as
tecnologias que utilizam plugins que devem ser instalados na
plataforma final, pois o desenvolvedor vai depender de uma empresa
terceira (nesse caso, o desenvolvedor do plugin para essa tecnologia),
que na maioria dos casos utiliza código proprietário, impedindo que o
próprio desenvolvedor desenvolva o plugin para essa outra plataforma,
e o jogo desenvolvido fica dependendo das plataformas em que o plugin
funciona.

No caso de jogos feitos com HTML5 e Javascript, a única dependência é
o navegador, que todos os dispositivos que acessam a internet possuem,
a única limitação é exigir um navegador com suporte as essas
tecnologias, o que acontece com muita frequência nos dispositivos que
possuem um poder de processamento considerável, que também é um dos
requisitos para o jogo ser executado sem que o jogador sofra com
lentidões.

\subsection{Fragmentação}

Quando um jogo é desenvolvido para rodar em diversas plataformas
(conhecidos como jogos multiplataforma) e muito provável que o
desenvolvedor terá que lidar com um problema de fragmentação, pois
para cada tipo de dispositivo o jogo terá que ser customizado de uma
maneira diferente dependendo da tecnologia que foi escolhida para o
desenvolvimento do mesmo.

Os desenvolvedores que usam as tecnologias que dependem de plugins para executar o
jogo não sofrem muito com a fragmentação, pois o plugin gerencia
grande parte das diferenças entre os dispositivos, permitindo ao
desenvolvedor se preocupar apenas com a lógica do jogo. Dependendo do
plugin utilizado, alguns poucos problemas podem chegar ao
desenvolvedor, como o tamanho de tela, por exemplo, pois ao jogar em
um dispositivo com uma tela menor (um celular, por exemplo), o jogo
deve se adaptar ao novo formato, e dependendo do estilo do jogo, o
programador terá que lidar com as diferençar e adaptar o jogo aos
vários tamanhos de tela.

Ao utilizar HTML5 o navegador se torna a plataforma de execução do
jogo, portanto, o jogo funcionará em todos os dispositivos que
suportarem o mesmo, com a vantagem de já estar instalado previamente
no dispositivo em questão.
Independente do navegador já estar instalado no sistema, o
desenvolvedor do jogo terá os mesmos problemas dos plugins, ou seja,
também terá de lidar com as diferenças no tamanhos das telas dos
dispositivos mobile.

Uma das desvantagens do HTML5 sobre as outras tecnologias é a versão
dos navegadores que pode ser diferente em cada dispositivo, assim
aumentando o problema de fragmentação. Esse problema é conhecido pelos
desenvolvedores web, e não é um grande problema quando definido um
requisito mínimo para se utilizar determinada aplicação.

\subsection{Recursos para desenvolvimento}

Cada tecnologia disponibiliza diferentes ferramentas para facilitar o
desenvolvimento em sua plataforma, para isso algumas utilizam meios
gráficos, como ferramentas WYSIWYG que facilitam o posicionamento de
objetos e outras apenas disponibilizam suas bibliotecas como um
framework para facilitar tarefas de código.

A adobe possui o Adobe Flash professional (http://www.adobe.com/br/products/flash.html),
que possibilita ao desenvolvedor um acesso rápido as ferramentas de cores
e desenhos, e facilita muito o posicionamento e a transição entre o
código e a amostragem do jogo, integrando os dois lados em uma única
plataforma, o que facilita o trabalho para o desenvolvedor
inexperiente.

O Unity editor (http://unity3d.com/unity/editor/) é o editor da engine Unity que possibilita toda a
criação do jogo utilizando a plataforma, permitindo, por exemplo, a
qualquer momento clicar no botão de iniciar e testar como está a sua
criação, emulando a plataforma que você deseja publicar.
Esse editor possui a facilidade de arrastar objetos e programá-los
facilmente, o que é totalmente necessário para um jogo 3D (que e o
foco do Unity).

Java não possui uma ferramenta diferente para o desenvolvimento de
jogos, apenas segue o mesmo padrão de desenvolvimento convencional
java e utiliza as bibliotecas de applet para preparar o jogo para
funcionar na web, ou seja, com uma das conhecidas IDEs (Eclipse por
exemplo) você consegue desenvolver um jogo casual em Java.
