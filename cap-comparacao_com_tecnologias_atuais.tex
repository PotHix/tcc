\section{Comparação com outras tecnologias web para jogos casuais}

%TODO: Adicionar uma enumeração sobre os comparativos.
Com base nas funcionalidades apresentadas sobre as tecnologias que
fazem parte da especificação do HTML5, será feita uma comparação com
as principais ferramentas utilizadas para o desenvolvimento de jogos
na data da escrita desse trabalho.

\subsection{Outras tecnologias utilizadas atualmente}

%TODO: Estruturar essa frase melhor. Até o "sobre"
Com exceção do HTML5, as ferramentas atuais para desenvolvimento de jogos para a internet se
utilizam de extensões que são instaladas nativamente no computador
para que possa ser uma plataforma na qual o jogo desenvolvido vai
executar sobre, para assim gerar uma plataforma mais consistente para
o desenvolvimento e ter um melhor acesso ao hardware do dispositivo em
questão. Essas tecnologias são Java, Flash e Unity.

\subsubsection{Java}

%TODO: Trocar "o java" por "a plataforma Java" ou remover pelo menos o artigo
Java foi uma das primeiras tecnologias que podia proporcionar um
ambiente favorável ao desenvolvimento de games portada para os
navegadores. Para que essa funcionalidade ficasse disponível para os
navegadores, a plataforma Java disponibilizou um segundo tipo de aplicação,
chamado applet.

Os applets são programas Java que podem ser inseridos em uma página de
internet \cite{boese2009introduction}. Esses programas são inseridos nas
páginas utilizando a tag \textit{<applet>} no documento HTML passando
o arquivo Java em um dos seus atributos. Com isso o programa Java é
baixado do servidor e executado na maquina do cliente pelo navegador,
que utiliza um ambiente de \textit{sandbox}, assim não permitindo
ao programa ter total acesso dentro da maquina do usuário.

Durante um bom período de tempo o Java foi a única tecnologia
aceitável para se colocar um jogo disponível via internet, até que
novas tecnologias começaram a surgir e o Java perdeu um pouco do seu
espaço como tecnologia para desenvolvimento de jogos para internet,
apesar dele ainda ser utilizado por alguns, principalmente no meio
acadêmico.

\subsubsection{Flash}

Flash é um produto criado pela Macromedia em 1996, com a finalidade de
enriquecer o conteúdo da internet, trazendo funcionalidades que não
era possível disponibilizar apenas com HTML na época de sua criação.
Com Flash, foi possível a criação de diversos sites com animações, e
foi o que originou as home pages animadas com um "Clique para entrar".

Em 2005 a Macromedia foi comprada pela Adobe e vem sendo mantido por
ela desde então, juntamente com todo o pacote de multimidia da
Macromedia.

A facilidade em gerar animações com o Flash trouxe a idéia de
utilizá-lo para a criação de jogos que funcionassem diretamente no
navegador, tendo em vista que plugin do Flash exporta um binário para ser
executado utilizando o Flash player instalado pelo usuário, a
performance pode ser bem melhor, pois o flash player tem acesso ao
hardware do dispositivo no qual ele está instalado.

Atualmente, Flash é a tecnologia mais utilizada para desenvolvimento de jogos para a
internet \cite{website:adobeflashleading}. Essa ferramenta possui muitas
bibliotecas feitas para facilitar o desenvolvimento de jogos, e por
possuir uma interface simples tanto para o designer como para o desenvolvedor,
ela vem mantendo seu legado ao longo dos anos.
Mais de 99 por cento dos computadores pessoais possuem o Flash Player
instalado \cite{website:adobeflashpenetration}. E esse é um dos grandes
motivos do sucesso do Flash, pois quebra um dos grandes problemas de
se desenvolver um jogo que necessita de dependencias externas para ser
executado.

\subsubsection{Unity}

%TODO: Falar um pouco sobre o surgimento da unity e a empresa
%responsável. Talvez dar algum exemplo que use Unity para web.
Conforme \citeonline{blackman2011beginning}, a \textit{engine} Unity 3D provê
um bom ponto de entrada para o mundo do desenvolvimento de jogos, pois
ela possui um bom balanceamento entre funcionalidades e um bom preço.

Unity é uma \textit{engine} relativamente nova para desenvolvimento de jogos e está
ganhando muito espaço no mercado devido a sua facilidade de uso e
aumento na popularidade dos seus plugins. Utilizando-a é possível
desenvolver apenas um jogo e exportá-lo para ser executado em
diferentes plataformas, entre elas o navegador, que utiliza um
plugin para executá-lo, e a quantidade de instalações do plugin já
ultrapassa 35 milhões \cite{blackman2011beginning}.

É possível exportar jogos para rodar diretamente no navegador, mas
esse não é o único caminho que a \textit{engine} Unity fornece, é possível exportar
para Windows, Mac e outros dispositivos que já possuem versões do
Unity rodando.

Apesar de possuir uma versão paga com muitas vantagens e novas
funcionalidades muito úteis para empresas de jogos, essa engine possui
também uma versão gratuita que permite ao desenvolvedor criar,
aprender e até mesmo vender os seus jogos sem precisar pagar nada, e
isso tem sido um dos grandes atrativos de novos desenvolvedores.

Essa engine vem crescendo bastante desde seu lançamento em 2009 e vem ganhando
cada vez mais novas pessoas na comunidade, que já possui mais de
400.000 (em abril de 2011 conforme
\citeonline{blackman2011beginning}), e vem disputando a fama de melhor plataforma
para desenvolvimento de jogos para o navegador.

\subsection{Performance}

%TODO: Falar o que são os tais recursos.
Performance é um assunto muito discutido no desenvolvimento de jogos
pois os jogos consomem muito recurso do dispositivo em que estão
executando, pois estão, entre outras coisas, em constante execução, fazendo
cálculos diversos para situar o jogador dentro do jogo, por exemplo.
Entre esses cálculos, dependendo do jogo, estão os cálculos de física dos
objetos da cena e do próprio jogador, transformação de imagens para
dar impressão de movimento ou sombra e outras coisas que exigem alto
processamento em um curto período de tempo.

As tecnologias que exigem a instalação de um software de terceiros
(Java, Unity e Flash) possuem uma performance melhor no
momento de escrita desse trabalho. O motivo desse ganho de performance
é a otimização da plataforma para utilização em um determinado
dispositivo, fazendo com que o jogo seja interpretado e executado
nativamente no sistema operacional, assim ganhando várias
vantagens como um processo separado no sistema, apenas executando um
jogo que foi previamente "compilado" especialmente para essa tecnologia.

O HTML5 juntamente com Javascript perde um pouco em performance, pois
o sistema operacional vê apenas um navegador como processo, e esse
navegador vai fazer a interface entre o jogo e o sistema operacional.
O navegador não é uma plataforma otimizada para jogos (como é o caso
dos plugins, que possuem apenas essa finalidade, ou são bem
direcionados para o tal), apesar de estar em constante evolução e
desenvolvimento para interpretar Javascript de uma maneira mais veloz,
a cada nova versão.

%TODO: "Se o desenvolvimento do jogo focar em performance"
% Mudar essa frase para dar uma idéia de que o foco em performance é
% necessário para que o jogo funcione bem em plataformas mobile.
O estado dos navegadores na data de escrita desse
trabalho suporta a execução de jogos jogos casuais simples, que exigem
um processamento aceitável do dispositivo, ou seja, jogos em duas
dimensões, com cálculos de física básica. Se o desenvolvimento do jogo
focar em performance, há uma grande possibilidade de que o mesmo
funcione sem grandes problemas e com uma performance aceitável em dispositivos
que possuem menor poder de processamento, como os celulares por exemplo.

\subsection{Portabilidade}

%TODO: Mencionar as plataformas que podem rodar jogos em HTML5, como
%android iPhone e etc.

Sempre que um jogo casual é desenvolvido, tem-se em mente que ele deve
abranger a maior quantidade de dispositivos possível, para assim
atingir um maior publico, e ser utilizado onde o jogador estiver (como
é o caso dos dispositivos móveis). Quando a tecnologia para o
desenvolvimento do jogo é escolhida esse quesito tem que ser
ponderado, pois ele vai ditar em qual tipo de plataforma o jogo em
questão será executado, pois dependendo da plataforma escolhida o jogo ficará preso
a um determinado fabricante ou dispositivo.

No caso de jogos desenvolvidos para a web, esse problema acontece com as
tecnologias que utilizam plugins que devem ser instalados na
plataforma final, pois o desenvolvedor vai depender de uma empresa
terceira (nesse caso, o desenvolvedor do plugin para essa tecnologia),
que na maioria dos casos utiliza código proprietário, impedindo que o
próprio desenvolvedor desenvolva o plugin para essa outra plataforma,
e o jogo desenvolvido se torna dependente das plataformas em que o plugin
funciona.

Um grande exemplo de problema pode ser notado com o Flash, que possui
plugin para os navegadores que funciona muito bem em sistemas Windows,
mas tem uma qualidade não muito boa em sistemas derivados de Unix como
o MacOs ou o Linux. Além desse baixa qualidade nos computadores, o
Flash não possui uma versão para os celulares da Apple, assim tirando
do desenvolvedor que escolher essa plataforma uma valiosa quantia de
jogadores que poderiam pagar pelo produto. O Flash possui uma versão
para celulares com o sistema operacional Android, mas a qualidade
também não é muito boa não trazendo muito sucesso para a plataforma em
dispositivos móveis.

No caso de jogos feitos com HTML5 e Javascript, a única dependência é
o navegador, que todos os dispositivos que acessam a internet possuem,
e a única limitação é exigir um navegador com suporte à essas
tecnologias, o que acontece com muita frequência nos dispositivos que
possuem um poder de processamento considerável, que também é um dos
requisitos para o jogo ser executado sem que o jogador sofra com
lentidões.

\subsection{Fragmentação}

Quando um jogo é desenvolvido para rodar em diversas plataformas
(conhecidos como jogos multiplataforma) é muito provável que o
desenvolvedor terá que lidar com um problema de fragmentação, pois
para cada tipo de dispositivo o jogo terá que ser customizado de uma
maneira diferente dependendo da tecnologia que foi escolhida para o
desenvolvimento do mesmo.

Os desenvolvedores que usam as tecnologias que dependem de plugins para executar o
jogo não sofrem muito com a fragmentação, pois o plugin gerencia
grande parte das diferenças entre os dispositivos, permitindo ao
desenvolvedor se preocupar apenas com a lógica do jogo. Dependendo do
plugin utilizado, alguns poucos problemas podem chegar ao
desenvolvedor, como o tamanho de tela, por exemplo, pois ao jogar em
um dispositivo com uma tela menor (um celular, por exemplo), o jogo
deve se adaptar ao novo formato, e dependendo das características do jogo, o
programador terá que lidar com as diferenças e adaptar o jogo aos
vários tamanhos de tela.

Ao utilizar HTML5 o navegador se torna a plataforma de execução do
jogo, portanto ele funcionará em todos os dispositivos que
suportarem o mesmo, com a vantagem de já estar instalado previamente
no dispositivo em questão.
Independente do navegador já estar instalado no sistema, o
desenvolvedor do jogo terá os mesmos problemas dos plugins, ou seja,
também terá de lidar com as diferenças no tamanhos das telas dos
dispositivos mobile.

Uma das desvantagens do HTML5 sobre as outras tecnologias é a versão
dos navegadores que pode ser diferente em cada dispositivo, assim
aumentando o problema de fragmentação. Esse problema é conhecido pelos
desenvolvedores web, e não é um grande problema quando definido um
requisito mínimo para se utilizar determinada aplicação. Um exemplo
simples desse problema é o jogo demo Quake III, que por ter seu código
aberto está sendo portado para rodar com WebGL diretamente no navegador
\cite{website:webglquake3}. Quando esse jogo é acessado por um
dispositivo que não possui suporte a WebGL o jogo não irá funcionar,
pois ele depende dessa plataforma para sua execução, portanto, quando
um jogo precisa de uma determinada funcionalidade para funcionar, e
essa funcionalidade ainda não é suportada por todos os navegadores
mais conhecidos, o usuário precisa ser informado do motivo dele não
conseguir executar o jogo ao acessá-lo.
O escopo desse trabalho são jogos casuais, assim excluiremos jogos
maiores como é o caso do exemplo acima, portanto para a maioria dos
casos é possível ter uma segunda alternativa para que o jogo seja
executado, mesmo que para isso a performance seja inferior.

\subsection{Recursos facilitadores para desenvolvimento}

Cada tecnologia disponibiliza diferentes ferramentas para facilitar o
desenvolvimento em sua plataforma, para isso algumas utilizam meios
gráficos, como ferramentas WYSIWYG que facilitam o posicionamento de
objetos e outras apenas disponibilizam suas bibliotecas como um
framework para facilitar tarefas de código.

A adobe possui o Adobe Flash professional \cite{website:adobeflash},
que possibilita ao desenvolvedor um acesso rápido às ferramentas de cores
e desenhos, e facilita muito o posicionamento e a transição entre o
código e a amostragem do jogo, integrando os dois lados em uma única
plataforma, o que facilita o trabalho para o desenvolvedor
inexperiente.

O Unity editor \cite{website:unity3d} é o editor da engine Unity que possibilita toda a
criação do jogo utilizando a plataforma, permitindo, por exemplo, a
qualquer momento clicar no botão de iniciar e testar como está a sua
criação, emulando a plataforma que você deseja publicar.
Esse editor possui a facilidade de arrastar objetos e programá-los
facilmente, o que é totalmente necessário para um jogo 3D (que e o
foco do Unity).

Java não possui uma ferramenta diferente para o desenvolvimento de
jogos, apenas segue o mesmo padrão de desenvolvimento convencional
Java e utiliza as bibliotecas de applet para preparar o jogo para
funcionar na web, ou seja, com uma das conhecidas IDEs (Eclipse por
exemplo) você consegue desenvolver um jogo casual em Java.

O HTML5 ainda não possui muitos recursos gráficos para facilitar no
desenvolvimento de jogos, mas muito já está sendo desenvolvido. Um dos
exemplos disso é a própria Adobe (criadora do Flash), que está
desenvolvendo uma ferramenta parecida com o Flash, que irá gerar
HTML5, Javascript e CSS para fazer animações, essa ferramenta é o
Adobe Edge \cite{website:adobeedge}, que já tem um preview disponível
gratuitamente para os desenvolvedores.

Outras empresas também estão trabalhando para oferecer engines para
desenvolvimento de jogos para o navegador, como é o caso da Rocket
Pack \cite{website:rocketpack}, que está desenvolvendo a Rocket Engine
\cite{website:rocketengine}, que ainda não suporta Canvas, mas já
exporta HTML5 utilizando elementos DOM para diagramar o conteúdo do
jogo no navegador.
