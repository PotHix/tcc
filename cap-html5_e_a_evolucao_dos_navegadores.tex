\section{HTML5 e a evolução dos navegadores}

Os navegadores evoluiram muito nos ultimos anos e com essa evolução
veio a possibilidade de utilização de tecnologias que integram
diretamente com o navegador, provendo muitas possibilidades de uso com
a vantagem de terceirizar o trabalho da implementação multi-plataforma
de uma aplicação, pois as empresas que desenvolvem os navagadores
terão todo esse trabalho pelo desenvolvedor. Essa se tornou uma das
grandes vantagens do HTML5, e com sua evolução constante, é possível
contruir diretamente no navegador muitas coisas que atualmente necessitam
de ferramentas de terceiros, apenas utilizando javascript para a
manipulação.

\subsection{Javascript e sua evolução}

Javascript é uma linguagem que foi criada em 1995 por Brendan Eich,
um engenheiro da netscape, e foi lançada no começo de 1996 com o
netscape 2 \cite{mdnjavascript}, sendo padronizada pela Ecma
international em 1997, lançando assim a primeira versão do ECMAScript.
De acordo com \cite{flanagan2006javascript}, Javascript é uma linguagem de programação
com capacidades de Orientação a objetos e que tem como uma das suas principais
utilizações a manipulação de objetos em navegadores, sendo que, nesse contexto
ela tem seu propósito extendido com objetos que permitem aos scripts uma interação com o usuário,
controlando o navegador e alterando o conteúdo do documento que aparece na janela do navegador.
Desde sua padronização, o Javascript o javascript vem sendo alterado e
recebendo algumas atualizações, sendo que a versão comummente
utilizada (e considerada estável) é a versão 3 padronizada em 1999.

\subsection{CSS e sua evolução}

CSS é uma linguagem para criação de folhas de estilo, que foi criada
para facilitar a criação de páginas para internet, comparada as tecnicas da época de
1990, o CSS fornece um controle bem maior sobre as páginas \cite{schmitt2009css}.
O CSS interage com um documento HTML, dando a ele uma melhor visualização, manipulando
fontes, cores e espaçamentos, assim deixando a diagramação e visualização do documento
separado da estrutura do mesmo, permitindo assim ter uma página semanticamente correta
e apenas manipulando a visualização utilizando CSS.
A versão mais utilizada atualmente, e considerada estável para utilização, é a versão 2
que se tornou padrão em maio de 1998, conforme \cite{zeldman2009designing}(pg. 248).
A versão 3 está em desenvolvimento e alguns dos navegadores atuais mais populares suportam
já suportam muitas de suas funcionalidades, apesar de o CSS3 estar em constante evolução.

\subsection{HTML e sua evolução}
%TODO: mudar esse texto, pois está igual a intrudução

HTML é uma linguagem de marcação de hipertexto que é utilizado como base para a internet
atual. O HTML mudou bastante desde sua versão inicial em 1991
\cite{powell2003html}, no início o HTML não era muito definido, e não tinha um padrão, até que que a
IETF (Internet Engeenering Task Force) começou a padronizá-lo em 1995, lançando a
primeira versão padronizada do HTML, que ficou conhecida como HTML 2.0.
Com o tempo o HTML foi se tornando mais padronizado, ganhando validações e novas integrações
(com CSS por exemplo), tornando-o mais versátil e adequado as novas necessidades da
web. Outra vantagem que a padronização trouxe, foi a facilidade de implementação pelos
navegadores, que agora implementavam com menos frequência o seu próprio estilo de marcação.
A ultima versão do HTML desenvolvida é o HTML5, que traz funcionalidades que dão um
olhar diferente para a internet, ou seja, ao invés de apenas se preocupar com texto
e formatação, o HTML5 visa atender as necessidades multimídia da internet, cobrindo
todos os espaços multimídia que tiveram que ser contornados com outras tecnologias,
como por exemplo vídeo, audio e desenho.
Com o crescimento do HTML, algumas coisas foram alteradas na parte de visualização,
removendo algumas tags que eram utilizadas para isso, como a tag font por exemplo.
Com a remoção dessas tags surgia a utilização de outra forma para
estilização de páginas, o CSS.

\subsection{A evolução dos navegadores}

Os navegadores nasceram muito antes da interface gráfica, segundo
\cite{robbins2006web}, e é possível notar sua constante evolução para
suprir as necessidades dos usuários e desenvolvedores. No início, os
navegadores eram apenas texto preto sobre um fundo cinza, e foi assim
nos primórdios da internet, enquanto o Mosaic e o Lynx eram os
browsers modo texto famosos. Em 1994 foi lançado o netscape 0.9, que
trouxe várias inovações para a internet, e dois anos depois a
Microsoft contra ataca com o Internet Explorer 3.0, com várias
funcionalidades novas, inclusive o suporte a folhas de estilo (e
introduzindo o CSS). e iniciando o que chamam de "A guerra dos
navegadores". Essa guerra foi travada entre a Microsoft e a Netscape
por muitos anos e trouxe muitas coisas boas e ruins para a internet. O
impacto positivo foi a aceleração do desenvolvimento para essas
plataforma, e o impacto negativo foi a grande falta de padrão para os
desenvolvedores, pois cada navegador implementava um formato diferente
de HTML e etc. A Microsoft venceu oficialmente a guerra dos navegadores quando a
Netscape foi comprada pela AOL, mas a Netscape deixou o código fonte
de seu navegador liberado como Open Source, e a Mozilla utilizou esse
código para começar o desenvolvimento do que futuramente seria o
Firefox.
Em 2005 foi lançada a primeira versão do Firefox e a Microsoft começa
a perder seu reinado absoluto com navegadores. O Firefox conseguiu uma
boa fatia do mercado de navegadores, e a Microsoft aos poucos vai
ficando para trás em termos de tecnologia e segurança com a versão 6
do internet explorer.
Com o passar dos anos, outros navegadores entram nessa disputa, entre
eles estão o Opera, que tem uma pequena parte do mercado, mas está em
constante desenvolvimento, e o Chrome do Google, que está entre os
mais rápidos e seguros navegadores da atualidade.
Novamente uma Guerra dos navegadores começa, mas dessa vez um pouco
mais controlada, pois existe um órgão regulamentador da internet que
todos os browsers procuram seguir e se adequar, ele se chama W3C
(World Wide Web Consortium), e seu papel nessa história é dizer o que
é padrão e o que não é, assim fazendo que a interface para os
desenvolvedores seja o mais parecida possível entre todos os
navegadores.
Ainda hoje cada navegador possui sua implementação de algumas
funcionalidades do HTML5 ou CSS3, e sua própria implementação do
interpretador de Javascript, o que faz alguns serem mais rápidos que
outros. E uma das grandes vantagens atuais, principalmente falando de
aplicações ricas, que tem um uso intenso de Javascript, é a velocidade
desse interpretador e o suporte as novas funcionalidades que o HTML5
propõe.
Os principais navegadores da atualidade são: Opera, Firefox, Internet
Explorer e Chrome. Todos estão implementando, cada um em seu ritmo, as
funcionalidades do HTML5, e estes serão usados como base para os
estudos desse trabalho.

\subsection{Jogos no navegador}

Lorem ipsum dolor sit amet, consectetur adipisicing elit, sed do eiusmod tempor incididunt ut labore et dolore magna aliqua. Ut enim ad minim veniam, quis nostrud exercitation ullamco laboris nisi ut aliquip ex ea commodo consequat. Duis aute irure dolor in reprehenderit in voluptate velit esse cillum dolore eu fugiat nulla pariatur.  Excepteur sint occaecat cupidatat non proident, sunt in culpa qui officia deserunt mollit anim id est laborum
