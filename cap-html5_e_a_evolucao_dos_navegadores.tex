\section{HTML5 e a evolução dos navegadores}

Os navegadores evoluiram muito nos ultimos anos e com essa evolução
veio a possibilidade de utilização de tecnologias que integram
diretamente com o navegador, provendo muitas possibilidades de uso com
a vantagem de terceirizar o trabalho da implementação multi-plataforma
de uma aplicação, pois as empresas que desenvolvem os navagadores
terão todo esse trabalho pelo desenvolvedor. Essa se tornou uma das
grandes vantagens do HTML5, e com sua evolução constante, é possível
contruir diretamente no navegador muitas coisas que atualmente necessitam
de ferramentas de terceiros, apenas utilizando javascript para a
manipulação.

\subsection{Javascript e sua evolução}

Javascript é uma linguagem que foi criada em 1995 por Brendan Eich,
um engenheiro da netscape, e foi lançada no começo de 1996 com o
netscape 2 \cite{mdnjavascript}, sendo padronizada pela Ecma
international em 1997, lançando assim a primeira versão do ECMAScript.
De acordo com \cite{flanagan2006javascript}, Javascript é uma linguagem de programação
com capacidades de Orientação a objetos e que tem como uma das suas principais
utilizações a manipulação de objetos em navegadores, sendo que, nesse contexto
ela tem seu propósito extendido com objetos que permitem aos scripts uma interação com o usuário,
controlando o navegador e alterando o conteúdo do documento que aparece na janela do navegador.
Desde sua padronização, o Javascript o javascript vem sendo alterado e
recebendo algumas atualizações, sendo que a versão comummente
utilizada (e considerada estável) é a versão 3 padronizada em 1999.

\subsection{CSS e sua evolução}
%TODO: mudar esse texto, pois está igual a intrudução

CSS é uma linguagem para criação de folhas de estilo, que interage com um documento
HTML, dando a ele uma melhor visualização, manipulando fontes, cores e espaçamentos.
A versão 2 do CSS se tornou padrão em maio de 1998, conforme
\cite{zeldman2009designing} (pg. 248), e atualmente está sendo
desenvolvida a versão 3, que já está parcialmente implementada nos navegadores modernos,
apesar de ainda estar em construção (assim como o HTML5).

\subsection{HTML e sua evolução}
%TODO: mudar esse texto, pois está igual a intrudução

HTML é uma linguagem de marcação de hipertexto que é utilizado como base para a internet
atual. O HTML mudou bastante desde sua versão inicial em 1991
\cite{powell2010html}, no início o HTML não era muito definido, e não tinha um padrão, até que que a
IETF (Internet Engeenering Task Force) começou a padronizá-lo em 1995, lançando a
primeira versão padronizada do HTML, que ficou conhecida como HTML 2.0.
Com o tempo o HTML foi se tornando mais padronizado, ganhando validações e novas integrações
(com CSS por exemplo), tornando-o mais versátil e adequado as novas necessidades da
web. Outra vantagem que a padronização trouxe, foi a facilidade de implementação pelos
navegadores, que agora implementavam com menos frequência o seu próprio estilo de marcação.
A ultima versão do HTML desenvolvida é o HTML5, que traz funcionalidades que dão um
olhar diferente para a internet, ou seja, ao invés de apenas se preocupar com texto
e formatação, o HTML5 visa atender as necessidades multimídia da internet, cobrindo
todos os espaços multimídia que tiveram que ser contornados com outras tecnologias,
como por exemplo vídeo, audio e desenho.
Com o crescimento do HTML, algumas coisas foram alteradas na parte de visualização,
removendo algumas tags que eram utilizadas para isso, como a tag font por exemplo.
Com a remoção dessas tags surgia a utilização de outra forma para
estilização de páginas, o CSS.

\subsection{A evolução dos navegadores}

Lorem ipsum dolor sit amet, consectetur adipisicing elit, sed do eiusmod tempor incididunt ut labore et dolore magna aliqua. Ut enim ad minim veniam, quis nostrud exercitation ullamco laboris nisi ut aliquip ex ea commodo consequat. Duis aute irure dolor in reprehenderit in voluptate velit esse cillum dolore eu fugiat nulla pariatur.  Excepteur sint occaecat cupidatat non proident, sunt in culpa qui officia deserunt mollit anim id est laborum

\subsection{Jogos no navegador}

Lorem ipsum dolor sit amet, consectetur adipisicing elit, sed do eiusmod tempor incididunt ut labore et dolore magna aliqua. Ut enim ad minim veniam, quis nostrud exercitation ullamco laboris nisi ut aliquip ex ea commodo consequat. Duis aute irure dolor in reprehenderit in voluptate velit esse cillum dolore eu fugiat nulla pariatur.  Excepteur sint occaecat cupidatat non proident, sunt in culpa qui officia deserunt mollit anim id est laborum
