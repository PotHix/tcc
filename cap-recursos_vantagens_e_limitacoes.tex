\section{Recursos disponíveis, vantagens e limitações}

As ferramentas atuais que suportam HTML5 possuem muitas vantagens,
principalmente em portabilidade, tirando do desenvolvedor a
responsabilidade de implemtação de código para várias plataformas, mas
com isso também introduz algumas limitações para que o jogo
desenvolvido possa funcionar em vários disposiivos e sistemas
operacionais. Esse capítulo visa mostrar as vantagens e desvantagens
de cada uma das ferramentas utilizadas no HTML5, o que pode ser usado
com tranquilidade e o que necessita de mais cautela no seu uso.

\subsection{Javascript}

Como dito no capítulo anterior, Javascript é a principal ferramenta
quando se vai trabalhar em um jogo com HTML5, pois é ela que vai
manipular as propriedades dos elementos para passar ao jogador a
impressão de movimento do protagonista do jogo.
Javascript é uma linguagem muito poderosa que suporta vários tipos de
sintaxes, e possui muitas vantagens de uma linguagem de alto nível,
como por exemplo coletor de lixo (garbage collector), e tipagem
dinamica, o que ajuda o desenvolvedor a se preocupar menos com tipos
de variáveis e elementos, e evita a grande preocupação com o
gerenciamento de memória.
Como desvantagem, o Javascript tende a expor o código do jogo em
quetão para a internet, pois, para que um navegador execute o jogo é
necessário que o mesmo possua o código, tendo em vista que Javascript
não é uma linguagem compilada, e sim interpretada. Uma das técnicas
utilizadas para evitar a distribuição do código para o usuário é a
ofoscação de código. Com essa técnica o seu código continua chegando
ao usuário, mas de uma maneira não legível, ou seja, será facilmente
entendida pelo navegador mas dificilmente entendida por um humano que
quer utilizar seu código fonte.
Uma das grandes preocupações de todos os desenvolvedores de jogos é a
performance. A área de jogos exige muito processamento para calcular
cada um dos passos que um jogador vai dar dentro de um jogo, como cada
objeto vai se comportar, calcular trajetórias de projéteis, calcular
iluminação dos objetos da cena, calcular a física dos objetos e outras
coisas, e para isso é necessário utilizar muito processamento do
dispositivo que estiver executando o mesmo. Sabemos que Javascript vai
ser interpretado em cima de um navegador, que por sua vez está sendo
executado pelo sistema operacional e dividindo memória e processamento
com outros programas no escalonador de processos, portanto, não
podemos esperar que um jogo que vai rodar no navegador tenha as mesmas
capacidades que um jogo que executa nativamente no sistema
operacional, com código compilado para a própria plataforma.
Tendo em mente a média de performance que é exigida por um jogo,
podemos levar em consideração que um jogo casual que utiliza apenas
coisas simples, como física básica, calculo de trajetórias e
movimentação de imagens é totalmente aceitável para ser executado em
dispositivos que possuem baixa performance, desde que bem executado.

\subsection{Navegadores}

Os navegadores são a parte fundamental para a execução do jogo, pois
ele é a plataforma de desenvolvimento que o desenvolvedor está visando,
ou seja, assim como muitas grandes empresas visam video games,
computadores ou celulares, como plataforma alvo de desenvolvimento, o
desenvolvedor deve ter em mente que está desenvolvendo para o
navegador, e estar ciente de suas vantagens, desvantagens e
limitações.

%TODO: Listar alguma funcionalidade do Diveintohtml5
Um dos grandes problemas de se trabalhar com navegadores é a
fragmentação, pois o desenvolvedor deve pensar em quais navegadores e
dispositivos o seu jogo estará disponível, levanvando em consideração
que cada usuário vai escolher seu navegador preferido, e que talvez
esse navegador não estará na ultima versão. Atualmente vários sites na
internet descrevem quais funcionalidades do HTML5 estão disponíveis em
qual navegador, um deles é o \cite{diveintohtml5}.
Esse problema acontece também no desenovimento para computadores e
video games, mas é um pouco mais simples de lidar, pois os jogos
simplesmente não são compatíveis, ou seja, se você tem um computador
que o hardware não é bom o suficiente para executar um jogo, ele será
barrado já na instalação, ou será executado mas será impossível de ser
jogado devido a sua lentidão. O mesmo acontece com os jogos no
navegador, caso o navegador não tenha suporte as funcionalidades que o
jogo precisa, ele simplesmente não irá ser executado. Enquanto o
desenvolvimento do jogo estiver sendo feito, o desenvolvedor deve ter
definido quais são as plataformas alvo do seu jogo, pois, apesar de
estar lidando com navegadores, que podem executar em uma grande
quantidade de plataformas, é nítido que não será possível ter a mesma
performance em todas elas, tendo em vista que um computador pessoal
possui muito mais capacidade de processamento do que um celular, por
exemplo.

O problema da fragmentação pode ser mais facilmente contornado quando
o desenvolvedor fixa algumas confugurações, como por exemplo a versão
dos navegadores suportados. A liberdade do desenvolvedor é total,
bastando apenas ter em mente o quanto de processamento será utilizado
e quais funcionalidades serão necessárias para a execução do jogo.

\subsection{Canvas}

O Canvas prove uma interface com resolução de dependencias para
desenhar coisas diretamente no navegador, e pode ser utilizado para
gerar gráficos, gráficos de jogos, e outras imagens para ser
visualizadas no momento da execução. \cite{w3ccanvas}

O canvas na web segue a mesma interface que é utilizada nas grandes
plataformas atuais. O conceito de canvas foi criado pela Apple para
utilizar no webkit do MacOS X \cite{lubbers2010pro}, criando uma API
para desenhar widgets e outras coisas em sua plataforma, e essa API
vem sendo padronizada e implementada em vários navegadores.

Essa tecnologia é fundamental para o desenvolvimento de jogos para
o navegador, pois permite ao desenvolvedor uma forma de exibir
objetos e manipulá-los da mesma forma que é feito com jogos
simples em duas dimensões para computadores.
Diferentemente do DOM, que cria vários objetos que podem ser
manipulados individualmente pelo navegador, o canvas quando criado
gera apenas um elemento no navegador, e todos os elementos criados
no nesse elemento (imagens, retangulos e etc) ficam invisíveis para acesso, e
não podem ser modificados, apenas recriados.

O canvas pode ser utilizado em várias camadas, gerando vários
elementos canvas dentro do navegador, e posicionando-os no mesmo
lugar, assim facilitando a manipulação de cenário e melhorando a
performance.

\subsection{WebSocket API}

WebSocket API define uma especificação que permite as páginas web
utilizar o protocolo de Sockets para utilizar uma comunicação em duas
vias com um servidor remoto \cite{w3cwebsockets}

Websockets possibilita uma redução de 500 para 1 ou até 1000 para 1
(em alguns casos) de cabeçalhos de HTTP desnecessários, e até 3 para 1
em latencia. \cite{lubbers2010pro}

\subsection{WebGL}

WebGL é uma API para gráficos 3D na web \cite{lubbers2010pro}.
WebGL é um mapeamento do OpenGL ES2 em Javascript \cite{lubbers2010pro}.
Algumas coisas legais que são possíveis de fazer com webgl \cite{webglmodel}.

Suporta shaders

baseado na tag canvas, mas utiliza o contexto 3d ao invés do 2d.


\subsection{Offline cache}
Offline cache, em sua maneira mais simplista, é uma lista apontando
para o HTML, CSS, Javascript, imagens e outros recursos que devem
estar disponíveis para o website \cite{pilgrim2010html5}.

Lorem ipsum dolor sit amet, consectetur adipisicing elit, sed do eiusmod tempor incididunt ut labore et dolore magna aliqua. Ut enim ad minim veniam, quis nostrud exercitation ullamco laboris nisi ut aliquip ex ea commodo consequat. Duis aute irure dolor in reprehenderit in voluptate velit esse cillum dolore eu fugiat nulla pariatur.  Excepteur sint occaecat cupidatat non proident, sunt in culpa qui officia deserunt mollit anim id est laborum

\subsection{SVG}
SVG é uma linguagem de marcação utilizada para descrever aplicações de gráficos
bi-dimensionais e imagens, e um conjunto de script relacionados a
interfaces graficas.\cite{w3csvg}

Lorem ipsum dolor sit amet, consectetur adipisicing elit, sed do eiusmod tempor incididunt ut labore et dolore magna aliqua. Ut enim ad minim veniam, quis nostrud exercitation ullamco laboris nisi ut aliquip ex ea commodo consequat. Duis aute irure dolor in reprehenderit in voluptate velit esse cillum dolore eu fugiat nulla pariatur.  Excepteur sint occaecat cupidatat non proident, sunt in culpa qui officia deserunt mollit anim id est laborum

\subsection{Local Storage}

Local Storage do HTML5 provê aos websites uma maneira de guardar
informações e recuperá-las depois, utilizando um conceito similar aos
cookies, mas já pensando em um grande modelo de dados.
\cite{pilgrim2010html5}.
Por muito tempo os desenvolvedores utilizaram cookies para guardar
informações sobre o website diretamente no navegador do usuário, e
essa tecnologia se provou muito ineficiente para uma quantidade um
pouco maior de dados, pois além de ser limitado para guardar informações,
os dados ficam trafegando pela rede em cada requisição, consumindo
banda, expondo os dados, e deixando as requisições mais lentas.
O Local storage foi criado para resolver esses problemas, dando uma
interface simples para o desenvolvedor guardar dados de sua
aplicação diretamente no navegador do usuário. Para guardar tais dados
o desenvolvedor utiliza um banco que armazana dados seguindo um
esquema chave-valor, ou seja, uma chave como índice, associado a um
determinado valor, que será convertido para texto quando adicionado.

%TODO: Dar um exemplo de uso de localstorage e falar de sua utilização em jogos casuais
