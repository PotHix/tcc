\section{Recursos disponíveis, vantagens e limitações}

As ferramentas atuais que suportam HTML5 possuem muitas vantagens,
principalmente em portabilidade, tirando do desenvolvedor a
responsabilidade de implemtação de código para várias plataformas, mas
com isso também introduz algumas limitações para que o jogo
desenvolvido possa funcionar em vários disposiivos e sistemas
operacionais. Esse capítulo visa mostrar as vantagens e desvantagens
de cada uma das ferramentas utilizadas no HTML5, o que pode ser usado
com tranquilidade e o que necessita de mais cautela no seu uso.

\subsection{Javascript}

Como dito no capítulo anterior, Javascript é a principal ferramenta
quando se vai trabalhar em um jogo com HTML5, pois é ela que vai
manipular as propriedades dos elementos para passar ao jogador a
impressão de movimento do protagonista do jogo.
Javascript é uma linguagem muito poderosa que suporta vários tipos de
sintaxes, e possui muitas vantagens de uma linguagem de alto nível,
como por exemplo coletor de lixo (garbage collector), e tipagem
dinamica, o que ajuda o desenvolvedor a se preocupar menos com tipos
de variáveis e elementos, e evita a grande preocupação com o
gerenciamento de memória.
Como desvantagem, o Javascript tende a expor o código do jogo em
quetão para a internet, pois, para que um navegador execute o jogo é
necessário que o mesmo possua o código, tendo em vista que Javascript
não é uma linguagem compilada, e sim interpretada. Uma das técnicas
utilizadas para evitar a distribuição do código para o usuário é a
ofoscação de código. Com essa técnica o seu código continua chegando
ao usuário, mas de uma maneira não legível, ou seja, será facilmente
entendida pelo navegador mas dificilmente entendida por um humano que
quer utilizar seu código fonte.
Uma das grandes preocupações de todos os desenvolvedores de jogos é a
performance. A área de jogos exige muito processamento para calcular
cada um dos passos que um jogador vai dar dentro de um jogo, como cada
objeto vai se comportar, calcular trajetórias de projéteis, calcular
iluminação dos objetos da cena, calcular a física dos objetos e outras
coisas, e para isso é necessário utilizar muito processamento do
dispositivo que estiver executando o mesmo. Sabemos que Javascript vai
ser interpretado em cima de um navegador, que por sua vez está sendo
executado pelo sistema operacional e dividindo memória e processamento
com outros programas no escalonador de processos, portanto, não
podemos esperar que um jogo que vai rodar no navegador tenha as mesmas
capacidades que um jogo que executa nativamente no sistema
operacional, com código compilado para a própria plataforma.
Tendo em mente a média de performance que é exigida por um jogo,
podemos levar em consideração que um jogo casual que utiliza apenas
coisas simples, como física básica, calculo de trajetórias e
movimentação de imagens é totalmente aceitável para ser executado em
dispositivos que possuem baixa performance, desde que bem executado.

\subsection{Navegadores}

Os navegadores são a parte fundamental para a execução do jogo, pois
ele é a plataforma de desenvolvimento que o desenvolvedor está visando,
ou seja, assim como muitas grandes empresas visam video games,
computadores ou celulares, como plataforma alvo de desenvolvimento, o
desenvolvedor deve ter em mente que está desenvolvendo para o
navegador, e estar ciente de suas vantagens, desvantagens e
limitações.

%TODO: Listar alguma funcionalidade do Diveintohtml5
Um dos grandes problemas de se trabalhar com navegadores é a
fragmentação, pois o desenvolvedor deve pensar em quais navegadores e
dispositivos o seu jogo estará disponível, levanvando em consideração
que cada usuário vai escolher seu navegador preferido, e que talvez
esse navegador não estará na ultima versão. Atualmente vários sites na
internet descrevem quais funcionalidades do HTML5 estão disponíveis em
qual navegador, um deles é o \cite{diveintohtml5}.
Esse problema acontece também no desenovimento para computadores e
video games, mas é um pouco mais simples de lidar, pois os jogos
simplesmente não são compatíveis, ou seja, se você tem um computador
que o hardware não é bom o suficiente para executar um jogo, ele será
barrado já na instalação, ou será executado mas será impossível de ser
jogado devido a sua lentidão. O mesmo acontece com os jogos no
navegador, caso o navegador não tenha suporte as funcionalidades que o
jogo precisa, ele simplesmente não irá ser executado. Enquanto o
desenvolvimento do jogo estiver sendo feito, o desenvolvedor deve ter
definido quais são as plataformas alvo do seu jogo, pois, apesar de
estar lidando com navegadores, que podem executar em uma grande
quantidade de plataformas, é nítido que não será possível ter a mesma
performance em todas elas, tendo em vista que um computador pessoal
possui muito mais capacidade de processamento do que um celular, por
exemplo.

O problema da fragmentação pode ser mais facilmente contornado quando
o desenvolvedor fixa algumas confugurações, como por exemplo a versão
dos navegadores suportados. A liberdade do desenvolvedor é total,
bastando apenas ter em mente o quanto de processamento será utilizado
e quais funcionalidades serão necessárias para a execução do jogo.

\subsection{Canvas}

Canvas é uma tecnologia que prove ao desenvolvedor uma forma fácil
para criar desenhos, importando imagens, criando figuras geométricas,
e manipulando-as de uma forma simples.
Essa tecnologia é a tecnologia fundamental para o desenvolvimento de
jogos para o navegador, pois permite ao desenvolvedor uma forma de
exibir objetos e manipulá-los da mesma forma que é feito com jogos
simples em duas dimensões para computadores.
% procurar mais informações na documentação do canvas do W3C

\subsection{WebSockets}

Lorem ipsum dolor sit amet, consectetur adipisicing elit, sed do eiusmod tempor incididunt ut labore et dolore magna aliqua. Ut enim ad minim veniam, quis nostrud exercitation ullamco laboris nisi ut aliquip ex ea commodo consequat. Duis aute irure dolor in reprehenderit in voluptate velit esse cillum dolore eu fugiat nulla pariatur.  Excepteur sint occaecat cupidatat non proident, sunt in culpa qui officia deserunt mollit anim id est laborum

\subsection{WebGL}

Lorem ipsum dolor sit amet, consectetur adipisicing elit, sed do eiusmod tempor incididunt ut labore et dolore magna aliqua. Ut enim ad minim veniam, quis nostrud exercitation ullamco laboris nisi ut aliquip ex ea commodo consequat. Duis aute irure dolor in reprehenderit in voluptate velit esse cillum dolore eu fugiat nulla pariatur.  Excepteur sint occaecat cupidatat non proident, sunt in culpa qui officia deserunt mollit anim id est laborum
\cite{webglmodel}

\subsection{Offline cache}

Lorem ipsum dolor sit amet, consectetur adipisicing elit, sed do eiusmod tempor incididunt ut labore et dolore magna aliqua. Ut enim ad minim veniam, quis nostrud exercitation ullamco laboris nisi ut aliquip ex ea commodo consequat. Duis aute irure dolor in reprehenderit in voluptate velit esse cillum dolore eu fugiat nulla pariatur.  Excepteur sint occaecat cupidatat non proident, sunt in culpa qui officia deserunt mollit anim id est laborum
