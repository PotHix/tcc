\section{Considerações finais}

Após analisar HTML5 como tecnologia para desenvolvimento de jogos para
a internet, utilizando os navegadores em seu atual estado de
desenvolvimento e comparando com as principais ferramentas utilizadas
para desenvolvimento até o momento, é possível se concluir que é
totalmente viável o desenvolvimento de um jogo casual simples
utilizando HTML5 e Javascript como tecnologia.

%TODO: Remover o "levando em consideração" e escrever um resumo de
%tudo mais ou menos.
Levando em consideração todas as vantagens e desvantagens da
tecnologia, é fácil perceber que os navegadores estão preparados para
receber jogos que exigem apenas cálculos relativamente
simples, como física básica e posicionamentos dos personagens, que é
praticamente o que é utilizado para se fazer um jogo simples. E aos
poucos os navegadores vão recebendo funcionalidades e otimizações para
lidar com jogos mais complexos, e também para melhorar a performance
dos jogos já desenvolvidos.

%TODO: Recapitular o que é o HTML5, todas as funcionalidades.
HTML5 é uma tecnologia nova que vem atender uma demanda conhecida da
internet mundial, e está assumindo esse papel muito bem, cada vez mais
ganhando publico e mostrando que realmente consegue atender a demanda
que foi criado para atender. Desenvolver jogos em HTML5 atualmente é
apostar no futuro da tecnologia, utilizando o que já está
desenvolvido, que é o suficiente para se fazer jogos simples, e
acompanhar a evolução do mesmo, desenvolvendo novas funcionalidade
conforme elas vão sendo desenvolvidas pelos navegadores, assim se
tornando referência na área de desenvolvimento de jogos com uma
tecnologia nova que irá revolucionar o mercado.

\subsection{Sugestões para pesquisas futuras}

Este trabalho visou mostrar as vantagens e desvantagens de se utilizar
HTML5 e Javascript para desenvolver um jogo casual para o navegador,
e há muito há ser pesquisado sobre esse assunto. Entre eles
estão:

%TODO: Arrumar essa frase para falar das vantagens e dificuldade de se
%obter fontes de lucro e etc.
Fontes de lucro, vantagens e dificuldades de um jogo em HTML5

%TODO: Explicar porque isso não faz parte do meu trabalho.
Performance e fragmentação no desenvolvimento para navegadores mobile
