\section[Considerações finais]{CONSIDERAÇÕES FINAIS}

Após analisar HTML5 como tecnologia para desenvolvimento de jogos para
a internet, utilizando os navegadores em seu atual estado de
desenvolvimento e comparando com as principais ferramentas utilizadas
para desenvolvimento até o momento, é possível concluir que é
totalmente viável o desenvolvimento de um jogo casual simples
utilizando HTML5 e Javascript como tecnologia.

Levando em consideração as funcionalidades do HTML5 como Canvas para o
desenho de objetos na tela, o Offline cache para manter um jogo
disponível sem a necessidade de uma conexão com a internet, o Local
Storage para guardar informações sem a nacessidade de trafegá-las em
cada requisição, os Websockets para fazer jogos multiplayer utilizando
uma conexão consistente, e todas as outras vantagens que foram
explicadas em mais detalhes durante o decorrer do trabalho,
é fácil perceber que os navegadores estão preparados para
receber jogos que exigem apenas cálculos relativamente
simples, como física básica e posicionamentos dos personagens, que é
praticamente o que é utilizado para se fazer um jogo simples. E aos
poucos os navegadores vão recebendo funcionalidades e otimizações para
lidar com jogos mais complexos, e também para melhorar a performance
dos jogos já desenvolvidos.

Como visto nos capítulos sobre as funcionalidades do HTML5, ele é um
conjunto de funcionalidades que visa melhorar o conteúdo
atual da internet sem a necessidade de ferramentas de terceiros, é uma
tecnologia nova que vem atender uma demanda conhecida da
internet mundial, e está assumindo esse papel muito bem, cada vez mais
ganhando publico e mostrando que realmente consegue atender a demanda
que foi criado para atender.

Desenvolver jogos em HTML5 atualmente é
apostar no futuro da tecnologia, pois como visto na comparação de
performance a evolução é rápida e apenas utilizando o que já está
desenvolvido é suficiente para fazer jogos simples. Como a tecnologia
é nova há uma grande oportunidade de acompanhar a evolução do mesmo,
utilizando novas funcionalidade conforme elas vão sendo desenvolvidas
pelos navegadores, assim se tornando referência na área de
desenvolvimento de jogos com uma das tecnologias que prometem a revolucionar o mercado.

Algumas empresas tem apostado nessa tecnologia, aproveitando
principalmente a vantagem de ter seu conteúdo disponível em
dispositivos móveis que não suportam outras tecnologias de terceiros
como Java, Flash e etc. Assim como mencionado durante a comparação das
tecnologias, até mesmo as empresas concorrentes estão investindo em
HTML5, criando novas ferramentas para ajudar
os desenvolvedores \cite{website:adobeedge}, e até mesmo fazendo
demonstrações e experiências para mostrar as capacidades do HTML5 como
plataforma para aplições RIA, também conhecidas como aplicações ricas
para internet.

Tendo em vista as vantagens e desvantagens do HTML5 como plataforma
para o desenvolvimento de jogos que foram descritas nesse trabalho é possível notar que os grandes
contratempos da plataforma estão na parte do desenvolvimento, e não no
produto final, e as ferramentas para facilitar tal desenvolvimento
estão sendo desenvolvidas aos poucos e disponibilizadas para os
desenvolvedores. A atualização dos navegadores pode ser um problema
para quem pretende utilizar essa tecnologia, pois apesar dos
navegadores estarem disponíveis gratuitamente para \textit{download}
muitos usuários não os atualizam permanecendo com versões
desatualizadas, mas como descrito anteriormente a grande maioria das empresas que desenvolvem
navegadores estão trabalhando para tornar essa atualização um processo
natural, sem que seja necessário a intervenção do usuário para a
atualização do navegador, assim diminuindo o problema.

Muitas empresas que já estão desenvolvendo jogos com HTML5
estão disponibilizando as ferramentas criadas por eles como
\textit{open source} e outras estão trabalhando em soluções pagas como
negócio, como é o caso da \citeonline{impactjs} mencionada
anteriormente, assim contribuindo para agilizar o desenvolvimento
de jogos utilizando essa tecnologia.

Baseando-se nos testes e comparações feitas é possível notar que o
HTML5 ainda não é uma ferramenta viável para receber jogos que
demandam alto processamento, mas é totalmente viável para jogos
casuais que exigem um processamento razoável. Esses tipos de jogos já possuem seu espaço no
mercado, e cada vez mais jogos maiores vão se tornarão realidade devido a constante evolução
do HTML5 e compatibilidade dos navegadores.

A tecnologia já oferece muitas vantagens atualmente e está em
constante evolução. Com o tempo novas funcionalidades serão
adicionadas, o suporte dos navegadores vai ficar cada vez melhor, e
com o investimento das empresas que desenvolvem os navegadores em manter
os usuários sempre com a ultima versão do navegador instalada, cada vez
mais será possível contar com uma plataforma preparada para receber
jogos casuais de qualidade.

\subsection{Sugestões para pesquisas futuras}

Este trabalho visou mostrar as vantagens e desvantagens de se utilizar
HTML5 e Javascript para desenvolver jogos casuais para o navegador,
para dar uma visão geral do que a plataforma é capaz de fazer, e quais
os problemas que devem ser evitados comparando-os com as tecnologias
que são utilizadas atualmente para tal finalidade.

O conteúdo desse trabalho não é focado em partes específicas do
desenvolvimento de jogos para poder dar uma visão geral da plataforma.
Há muito a ser pesquisado sobre as várias áreas mencionadas nesse
trabalho, Entre os vários assuntos que podem estender esse trabalho
estão:

\begin{itemize}
    \item Como obter lucro com desenvolvimento de jogos em HTML5, quais as vantagens
    e desvantagens que a plataforma pode oferecer para essa
    finalidade. Este trabalho não cobre esse tema pois muito deve ser
    pesquisado sobre propaganda e como aplica-la de uma maneira a ser
    melhor aproveitada em um jogo casual, além do estudo dos melhores
    meios de venda de jogos para a internet e a disponibilização de
    serviços para conseguir tal renda do jogador.

    \item Desenvolvimento de jogos com HTML5 focado em dispositivos móveis, como
    obter performance, utilizar uma banda limitada e menor consumo de
    bateria. Este trabalho não cobre esse tema pois isso requer uma maior
    pesquisa sobre o hardware dos dispositivos móveis e suas limitações,
    estudos de performance a fundo para mostrar como conseguir um menor
    custo de processamento, assim reduzindo o consumo de bateria, entre
    outras coisas que fariam com que esse trabalho tomasse um enfoque
    diferente.

    \item Modelo de negócios para dar subsídio a jogos e aplicações
    que são executadas no navegador dos celulares. Apesar de ser viável
    a criação de jogos e aplicações simples que funcionam no navegador
    dos celulares eles ainda não são muito bem difundidos, sendo que a
    grande preferencia dos usuários ainda está nas aplicações nativas.
    Alguma forma de divulgação deve ser desenvolvida para mostrar aos
    usuários as facilidades de ter um jogo sendo executada diretamente
    no navegador do celular, o que faria uma integração natural quando
    o mesmo jogo fosse jogado em outro dispositivo. Esse trabalho não
    cobre esse tópico pois muito deve ser pesquisado com relação as
    escolhas dos usuários e na usabilidade do celulares e dos
    navegadores disponíveis para os mesmos, fugindo assim do tema
    principal.
\end{itemize}
