\section[Considerações finais]{CONSIDERAÇÕES FINAIS}

Após analisar HTML5 como tecnologia para desenvolvimento de jogos para
a internet, utilizando os navegadores em seu atual estado de
desenvolvimento e comparando com as principais ferramentas utilizadas
para desenvolvimento até o momento, é possível se concluir que é
totalmente viável o desenvolvimento de um jogo casual simples
utilizando HTML5 e Javascript como tecnologia.

Levando em consideração as funcionalidades do HTML5 como Canvas para o
desenho de objetos na tela, o Offline cache para manter um jogo
disponível sem a necessidade de uma conexão com a internet, o Local
Storage para guardar informações sem a nacessidade de trafegá-las em
cada requisição, os Websockets para fazer jogos multiplayer utilizando
uma conexão consistente, e todas as outras vantagens que foram
explicadas em mais detalhes durante o decorrer do trabalho,
é fácil perceber que os navegadores estão preparados para
receber jogos que exigem apenas cálculos relativamente
simples, como física básica e posicionamentos dos personagens, que é
praticamente o que é utilizado para se fazer um jogo simples. E aos
poucos os navegadores vão recebendo funcionalidades e otimizações para
lidar com jogos mais complexos, e também para melhorar a performance
dos jogos já desenvolvidos.

HTML5 é um conjunto de funcionalidades que visa melhorar o conteúdo
atual da internet sem a necessidade de ferramentas de terceiros, é uma
tecnologia nova que vem atender uma demanda conhecida da
internet mundial, e está assumindo esse papel muito bem, cada vez mais
ganhando publico e mostrando que realmente consegue atender a demanda
que foi criado para atender.

Desenvolver jogos em HTML5 atualmente é
apostar no futuro da tecnologia, utilizando o que já está
desenvolvido, que é o suficiente para se fazer jogos simples, e
acompanhar a evolução do mesmo, desenvolvendo novas funcionalidade
conforme elas vão sendo desenvolvidas pelos navegadores, assim se
tornando referência na área de desenvolvimento de jogos com uma
tecnologia nova que irá revolucionar o mercado.

Algumas empresas tem apostado nessa tecnologia, aproveitando
principalmente a vantagem de ter seu conteúdo disponível em
dispositivos móveis que não suportam outras tecnologias de terceiros
como Java, Flash e etc. Aos poucos a própria Adobe está investindo em
HTML5, criando novas ferramentas para ajudar
os desenvolvedores \cite{website:adobeedge}, e até mesmo fazendo
demonstrações e experiências para mostrar as capacidades do HTML5 como
plataforma para aplições RIA, também conhecidas como aplicações ricas
para internet.

\subsection{Vantagens do HTML5}
Após as comparações feitas com as tecnologias que são utilizadas para
o desenvolvimento de jogos para a web na data de escrita desse
trabalho é possível chegar a uma conclusão sobre quais são
as vantagens do HTML5 atualmente.

Uma das vantagens claras que atrapalham muito os desenvolvedores de
jogos atualmente é a portabilidade, pois os desenvolvedores muitas
vezes precisam traduzir seu código para um grande quantidade de
plataformas para poder atingir uma quantidade maior de jogadores. O
problema com esse modelo é a fragmentação que isso gera, pois o mesmo
código estará dividido entre várias plataformas, com várias linguagens
de programação e provavelmente várias equipes de trabalho para manter
o jogo funcionando e corrigir cada problema que apareça. Quando um
problema é detectado na parte padrão do jogo todas as equipes devem
ser informadas e todos devem fazer determinada alteração e publicar o
jogo novamente para a plataforma em que o jogo está executando.

Utilizando HTML5 esse problema é resolvido com facilidade, pois caso
um problema seja encontrado na parte padrão do jogo, ele vai ser
corrigido no jogo principal e enviado para o ambiente de produção, com
isso todos os jogadores terão o jogo atualizado assim que acessarem o
jogo em um navegador com conexão com a internet. Isso facilita muito a
distribuição dos jogos, apenas cabendo ao desenvolvedor tratar os
tamanhos de telas e dispositivos de entrada no seu jogo.

A grande maioria dos jogadores possuem acesso a internet e geralmente
estão conectados em suas redes sociais ou fazendo tarefas corriqueiras
tanto em casa como em lugares monótonos como uma fila de banco por
exemplo. Se o jogo foi desenvolvido com HTML5, o jogador muito
provavelmente vai ter esse jogo disponível onde ele estiver, sem
depender de uma determinada plataforma (como um computador de mesa por
exemplo) para que continue seu jogo ou pelo menos o distraia num
momento monótono que precisa ser enfrentado. Com o crescimento do uso
de celulares e tablets o uso de tecnologias que façam o produto final
ficar disponível para essas plataformas também se faz cada vez mais
necessária para conseguir um publico maior.

Uma outra vantagem é dar ao jogador a liberdade de utilizar o
navegador que mais lhe agrada para poder rodar o jogo desenvolvido,
deixando apenas informado para o jogador quais são as funcionalidades
necessárias para que o jogo seja executado, e algunas recomendações
sobre navegadores. Apesar do jogo provavelmente funcionar bem sobre o
navegador padrão do usuário (se ele estiver devidamente atualizado
para a ultima versão disponível), alguns jogadores podem não gostar de
utilizá-lo, nesse caso ele pode simplesmente utilizar
o seu navegador preferido, que se estiver em sua ultima
versão e possuir suporte as funcionalidades mais novas do HTML5 será
capaz de executar o jogo sem maiores problemas. Outra vantagem disso é
dar a oportunidade do jogador testar qual o navegador que possui uma
maior performance na plataforma que ele está utilizando, dando ao
jogador uma grande gama de opções para que ele possa escolher a opção
que mais lhe convém.

\subsection{Desvantagens do HTML5}
Assim como toda tecnologia o HTML5 também possui suas desvantagens
sobre seus concorrentes, e muitas vezes as vantagens se tornam
desvantagens dependendo do ponto de vista do desenvolvedor.

A vantagem de se ter apenas um código para todas as plataformas é
considerada desvantagem por alguns desenvolvedores, pois dessa maneira
muitas condições devem ser tratadas para que o jogo funcione bem em
todas as plataformas, e isso torna o código difícil de ser mantido
dependendo de como o desenvolvedor organizá-lo.
Esse mesmo problema acontece com desenvolvedores de aplicações para o
sistema operacional Android, que precisam tratar as diferentes
configurações de \textit{hardware} de cada dispositivo, suas
velocidades de processamento, tamanho de tela e etc. Alguns
desenvolvedores (como é o caso da Rovio, criadora do jogo Angry Birds)
decidiram fazer diversos jogos para diversos aparelhos para evitar
esse tipo de fragmentação.

Por ser uma tecnologia nova ainda não existem muitas ferramentas para
auxiliar o desenvolvedor no desenvolvimento de jogos, exigindo do
mesmo um maior contato com o código e a execução de algumas tarefas
manualmente. Um exemplo disso é a criação de sprites para um jogo 2D,
que precisa ser feita em um editor de imagens padrão e mapeada
manualmente para o código, para que seja possível saber em qual parte
da imagem está cada sprite do jogo.
Atualmente há algumas ferramentas que estão sendo criadas pelos
próprios desenvolvedores para facilitar essa tarefa, assim como as
outras que ainda fazem falta no HTML5, ou seja, é uma questão de tempo
para que essa tecnologia receba cada vez mais boas ferramentas para
facilitar esses trabalhos manuais.

Uma desvantagem que não pode ser controlada pelo desenvolvedor são as
atitudes do usuário com a atualização do navegador, sendo assim, caso
o usuário possua um navegador desatualizado o jogo provavelmente não
funcionará como deveria, e dependendo das funcionalidades que ele
necessita ele não chegará a ser executado. Para reduzir esse problema,
as empresas que desenvolvem os navegadores estão adotando uma forma
transparente de atualizações nos navegadores para que o usuário não
precise se preocupar em manter seu navegador atualizado todo o tempo,
deixando essa preocupação com as empresas que os desenvolvem.
A primeira empresa a utilizar o sistema de atualização automática do
navegador foi a Google, que automaticamente atualiza o Google Chrome
de seus usuários, assim deixando-os sempre com a ultima versão estável
do navegador. Essa configuração pode ser desabilitada para os casos em
que uma determinada versão precisa ser utilizada, mas para o caso
geral dos usuários domésticos essa é uma ótima funcionalidade, pois dá
ao desenvolvedor web uma maior gama de possibilidades para os usuários
desse navegador, podendo assumir que ele está utilizando a ultima
versão estável.
A Mozilla após ver o bom modelo de atualização do Google Chrome
decidiu também utilizá-lo no seu navegador, o Firefox. Os usuários
também começam a receber uma nova versão a cada quatro meses, e sempre
estarão utilizando as novas versões do navegador, além de fazer várias
campanhas para que as pessoas que ainda utilizam as versões antigas do navegador (que ainda
não possuía suporte a atualização automática) atualizem o mesmo para
se beneficiar dessa nova funcionalidade.

\subsection{Avaliação das vantagens e desvantagens}

Tendo em vista as vantagens e desvantagens do HTML5 como plataforma
para o desenvolvimento de jogos é possível notar que os grandes
contratempos da plataforma estão na parte do desenvolvimento, e não no
produto final, e as ferramentas para facilitar tal desenvolvimento
estão sendo desenvolvidas aos poucos e disponibilizadas para os
desenvolvedores. Muitas empresas que já estão desenvolvendo jogos com HTML5
estão disponibilizando as ferramentas criadas por eles como open source
e outras estão trabalhando em soluções pagas como negócio.
O mercado de desenvolvimento de jogos com HTML5 já é real e está se
desenvolvendo a cada dia.

A tecnologia já oferece muitas vantagens atualmente e está em
constante evolução. Com o tempo novas funcionalidades serão
adicionadas, o suporte dos navegadores vai ficar cada vez melhor, e
com o investimento das empresas que desenvolvem os navegadores em manter
os usuários sempre com a ultima versão do navegador instalada, cada vez
mais será possível contar com uma plataforma preparada para receber
jogos casuais de qualidade.

\subsection{Sugestões para pesquisas futuras}

Este trabalho visou mostrar as vantagens e desvantagens de se utilizar
HTML5 e Javascript para desenvolver jogos casuais para o navegador,
para dar uma visão geral do que a plataforma é capaz de fazer, e quais
os problemas que devem ser evitados comparando-os com as tecnologias
que são utilizadas atualmente para tal finalidade.

O conteúdo desse trabalho não é focado em partes específicas do
desenvolvimento de jogos para poder dar uma visão geral da plataforma.
Há muito a ser pesquisado sobre as várias áreas mencionadas nesse
trabalho, Entre os vários assuntos que podem estender esse trabalho
estão:

\begin{itemize}
    \item Como obter lucro com desenvolvimento de jogos em HTML5, quais as vantagens
    e desvantagens que a plataforma pode oferecer para essa
    finalidade. Este trabalho não cobre esse tema pois muito deve ser
    pesquisado sobre propaganda e como aplica-la de uma maneira a ser
    melhor aproveitada em um jogo casual, além do estudo dos melhores
    meios de venda de jogos para a internet e a disponibilização de
    serviços para conseguir tal renda do jogador.

    \item Desenvolvimento de jogos com HTML5 focado em dispositivos móveis, como
    obter performance, utilizar uma banda limitada e menor consumo de
    bateria. Este trabalho não cobre esse tema pois isso requer uma maior
    pesquisa sobre o hardware dos dispositivos móveis e suas limitações,
    estudos de performance a fundo para mostrar como conseguir um menor
    custo de processamento, assim reduzindo o consumo de bateria, entre
    outras coisas que fariam com que esse trabalho tomasse um enfoque
    diferente.

    \item Modelo de negócios para dar subsídio a jogos e aplicações
    que são executadas no navegador dos celulares. Apesar de ser viável
    a criação de jogos e aplicações simples que funcionam no navegador
    dos celulares eles ainda não são muito bem difundidos, sendo que a
    grande preferencia dos usuários ainda está nas aplicações nativas.
    Alguma forma de divulgação deve ser desenvolvida para mostrar aos
    usuários as facilidades de ter um jogo sendo executada diretamente
    no navegador do celular, o que faria uma integração natural quando
    o mesmo jogo fosse jogado em outro dispositivo. Esse trabalho não
    cobre esse tópico pois muito deve ser pesquisado com relação as
    escolhas dos usuários e na usabilidade do celulares e dos
    navegadores disponíveis para os mesmos, fugindo assim do tema
    principal.
\end{itemize}
