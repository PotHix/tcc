\section{Considerações finais}
%TODO: Escrever um pouco mais sobre esse tópico

Após analisar HTML5 como tecnologia para desenvolvimento de jogos para
a internet, utilizando os navegadores em seu atual estado de
desenvolvimento e comparando com as principais ferramentas utilizadas
para desenvolvimento até o momento, é possível se concluir que é
totalmente viável o desenvolvimento de um jogo casual simples
utilizando HTML5 e Javascript como tecnologia.

Levando em consideração as funcionalidades do HTML5 como Canvas para o
desenho de objetos na tela, o Offline cache para manter um jogo
disponível sem a necessidade de uma conexão com a internet, o Local
Storage para guardar informações sem a nacessidade de trafegá-las em
cada requisição, os Websockets para fazer jogos multiplayer utilizando
uma conexão consistente, e todas as outras vantagens que foram
explicadas em mais detalhes durante o decorrer do trabalho,
é fácil perceber que os navegadores estão preparados para
receber jogos que exigem apenas cálculos relativamente
simples, como física básica e posicionamentos dos personagens, que é
praticamente o que é utilizado para se fazer um jogo simples. E aos
poucos os navegadores vão recebendo funcionalidades e otimizações para
lidar com jogos mais complexos, e também para melhorar a performance
dos jogos já desenvolvidos.

HTML5 é um conjunto de funcionalidades que visa melhorar o conteúdo
atual da internet sem a necessidade de ferramentas de terceiros, é uma
tecnologia nova que vem atender uma demanda conhecida da
internet mundial, e está assumindo esse papel muito bem, cada vez mais
ganhando publico e mostrando que realmente consegue atender a demanda
que foi criado para atender.

Desenvolver jogos em HTML5 atualmente é
apostar no futuro da tecnologia, utilizando o que já está
desenvolvido, que é o suficiente para se fazer jogos simples, e
acompanhar a evolução do mesmo, desenvolvendo novas funcionalidade
conforme elas vão sendo desenvolvidas pelos navegadores, assim se
tornando referência na área de desenvolvimento de jogos com uma
tecnologia nova que irá revolucionar o mercado.

Algumas empresas tem apostado nessa tecnologia, aproveitando
principalmente a vantagem de ter seu conteúdo disponível em
dispositivos móveis que não suportam outras tecnolovias de terceiros
como Java, Flash e etc. Aos poucos a própria Adobe está investindo em
HTML5, criando novas ferramentas para ajudar
os desenvolvedores \cite{website:adobeedge}, e até mesmo fazendo
demonstrações e experiências para mostrar as capacidades do HTML5 como
plataforma para aplições RIA, também conhecidas como aplicações ricas
para internet.

\subsection{Vantagens do HTML5}
Após as comparações é possível chegar a uma conclusão sobre quais são
as vantagens do HTML5 sobre as tecnologias que são utilizadas
atualmente.

Uma das vantagens claras que atrapalham muito os desenvolvedores de
jogos atualmente é a portabilidade, pois os desenvolvedores muitas
vezes precisam traduzir seu código para um grande quantidade de
plataformas para poder atingir uma quantidade maior de jogadores. O
problema com esse modelo é a fragmentação que isso gera, pois o mesmo
código estará dividido entre várias plataformas, com várias linguagens
de programação e provavelmente várias equipes de trabalho para manter
o jogo funcionando e corrigir cada problema que apareça. Quando um
problema é detectador na parte padrão do jogo todas as equipes devem
ser informadas e todos devem fazer determinada alteração e publicar o
jogo novamente para a plataforma que o jogo está executando.

Utilizando HTML5 esse problema é resolvido com facilidade, pois caso
um problema seja encontrado na parte padrão do jogo, ele vai ser
corrigido no jogo principal e enviado para o ambiente de produção, com
isso todos os jogadores terão o jogo atualizado assim que acessarem o
jogo em um navegador com conexão com a internet. Isso facilita muito a
distribuição dos jogos, apenas cabendo ao desenvolvedor tratar os
tamanhos de telas e dispositivos de entrada no seu jogo.

\subsection{Desvantagens do HTML5}
Assim como toda tecnologia o HTML5 também possui suas desvantagens
sobre seus concorrentes, e muitas vezes as vantagens se tornam
desvantagens dependendo do ponto de vista do desenvolvedor.

A vantagem de se ter apenas um código para todas as plataformas é
considerada desvantagem por alguns desenvolvedores, pois dessa maneira
muitas condições devem ser tratadas para que o jogo funcione bem em
todas as plataformas, e isso torna o código difícil de ser mantido
dependendo de como o desenvolvedor organizá-lo.
Esse mesmo problema acontece com desenvolvedores de aplicações para o
sistema operacional Android, que precisam tratar as diferentes
configurações de \textit{hardware} de cada dispositivo, suas
velocidades de processamento, tamanho de tela e etc. Alguns
desenvolvedores (como é o caso da Rovio, criadora do jogo Angry Birds)
decidiram fazer diversos jogos para diversos aparelhos para evitar
esse tipo de fragmentação.

Por ser uma tecnologia nova ainda não existem muitas ferramentas para
auxiliar o desenvolvedor no desenvolvimento de jogos, exigindo do
mesmo um maior contato com o código e a execução de algumas tarefas
manualmente. Um exemplo disso é a criação de sprites para um jogo 2D,
que precisa ser feita em um editor de imagens padrão e mapeada
manualmente para o código, para que seja possível saber em qual parte
da imagem está cada sprite do jogo.
Atualmente há algumas ferramentas que estão sendo criadas pelos
próprios desenvolvedores para facilitar essa tarefa, assim como as
outras que ainda fazem falta no HTML5, ou seja, é uma questão de tempo
para que essa tecnologia receba cada vez mais boas ferramentas para
facilitar esses trabalhos manuais.

\subsection{Sugestões para pesquisas futuras}

Este trabalho visou mostrar as vantagens e desvantagens de se utilizar
HTML5 e Javascript para desenvolver jogos casuais para o navegador,
para dar uma visão geral do que a plataforma é capaz de fazer, e quais
os problemas que devem ser evitados comparando-os com as tecnologias
que são utilizadas atualmente para tal finalidade.

O conteúdo desse trabalho não é focado em partes específicas do
desenvolvimento de jogos para poder dar uma visão geral da plataforma.
Há muito há ser pesquisado sobre as várias áreas mencionadas nesse
trabalho, Entre os vários assuntos que podem extender esse trabalho
estão:

Como obter lucro com desenvolvimento de jogos em HTML5, quais as vantagens
e desvantagens que a plataforma pode oferecer para essa
finalidade. Este trabalho não cobre esse tema pois muito deve ser
pesquisado sobre propaganda e como aplica-la de uma maneira a ser
melhor aproveitada em um jogo casual, além do estudo dos melhores
meios de venda de jogos para a internet e a disponibilização de
serviços para conseguir tal renda do jogador.

Desenvolvimento de jogos com HTML5 focado em dispositivos móveis, como
obter performance, utilizar uma banda limitada e menor consumo de
bateria. Este trabalho não cobre esse tema pois isso requer uma maior
pesquisa sobre o hardware dos dispositivos móveis e suas limitações,
estudos de performance a fundo para mostrar como conseguir um menor
custo de processamento, assim reduzindo o consumo de bateria, entre
outras coisas que fariam com que esse trabalho tomasse um enfoque
diferente.
