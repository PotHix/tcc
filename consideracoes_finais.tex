\section{Considerações finais}

Após analisar HTML5 como tecnologia para desenvolvimento de jogos para
a internet, utilizando os navegadores em seu atual estado de
desenvolvimento e comparando com as principais ferramentas utilizadas
para desenvolvimento até o momento, é possível se concluir que é
totalmente viável o desenvolvimento de um jogo casual simples
utilizando HTML5 e Javascript como tecnologia.

Levando em consideração as funcionalidades do HTML5 como Canvas para o
desenho de objetos na tela, o Offline cache para manter um jogo
disponível sem a necessidade de uma conexão com a internet, o Local
Storage para guardar informações sem a nacessidade de trafegá-las em
cada requisição, os Websockets para fazer jogos multiplayer utilizando
uma conexão consistente, e todas as outras vantagens que foram
explicadas em mais detalhes durante o decorrer do trabalho,
é fácil perceber que os navegadores estão preparados para
receber jogos que exigem apenas cálculos relativamente
simples, como física básica e posicionamentos dos personagens, que é
praticamente o que é utilizado para se fazer um jogo simples. E aos
poucos os navegadores vão recebendo funcionalidades e otimizações para
lidar com jogos mais complexos, e também para melhorar a performance
dos jogos já desenvolvidos.

HTML5 é um conjunto de funcionalidades que visa melhorar o conteúdo
atual da internet sem a necessidade de ferramentas de terceiros, é uma
tecnologia nova que vem atender uma demanda conhecida da
internet mundial, e está assumindo esse papel muito bem, cada vez mais
ganhando publico e mostrando que realmente consegue atender a demanda
que foi criado para atender. Desenvolver jogos em HTML5 atualmente é
apostar no futuro da tecnologia, utilizando o que já está
desenvolvido, que é o suficiente para se fazer jogos simples, e
acompanhar a evolução do mesmo, desenvolvendo novas funcionalidade
conforme elas vão sendo desenvolvidas pelos navegadores, assim se
tornando referência na área de desenvolvimento de jogos com uma
tecnologia nova que irá revolucionar o mercado.

\subsection{Sugestões para pesquisas futuras}

Este trabalho visou mostrar as vantagens e desvantagens de se utilizar
HTML5 e Javascript para desenvolver um jogo casual para o navegador,
e há muito há ser pesquisado sobre esse assunto. Entre eles
estão:

Como obter lucro com desenvolvendo jogos em HTML5, quais as vantagens
e desvantagens que a plataforma pode lhe oferecer para essa
finalidade. Este trabalho não cobre esse tema pois muito deve ser
pesquisado sobre propaganda e como aplica-la de uma maneira a ser
melhor aproveitada em um jogo casual.

Desenvolvimento de jogos com HTML5 focado em dispositivos móveis, como
obter performance, utilizar uma banda limitada e menor consumo de
bateria. Este trabalho não cobre esse tema pois isso requer uma maior
pesquisa sobre o hardware dos dispositivos móveis e suas limitações,
estudos de performance a fundo para mostrar como conseguir um menor
custo de processamento, assim reduzindo o consumo de bateria, entre
outras coisas que fariam com que esse trabalho tomasse um enfoque
diferente.
