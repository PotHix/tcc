\addcontentsline{toc}{section}{Glossário}

\section*{GLOSSÁRIO}

\begin{description}
\item[3G ] Terceira geração de tecnologias de acesso móvel, hoje
bastante conhecida por fornecer conexão com a internet.
\item[Android ] Sistema operacional para dispositivos móveis criado
pelo Google, possui um navegador que suporta vários recursos do HTML5.
\item[DOM ] Document Object Model, ou modelo de objeto de documentos,
é uma especificação da W3C onde é possível editar a estrutura
dinamicamente. Essa especificação possui uma interface padrão para
acesso aos elementos do documento, facilitando a manipulação dos
elementos individualmente.
\item[Driver ] Software que provê ao sistema operacional uma forma
fácil de lidar com um determinado hardware.
\item[Eclipse ] É uma ferramenta para desenvolvimento de software, uma
IDE (Integrated Development Environment, ou Ambiente de
desenvolvimento integrado).
\item[IETF ] Internet Engeenering Task Force
\item[iPhone ] Dispositivo móvel criado pela Apple, possui o iOS como
sistema operacional e um navegador que suporta vários recursos do HTML5.
\item[Engine ] Software de auxílio ao desenvolvimento de jogos. Provê
uma boa interatividade com o desenvolvedor, utilizando uma interface
gráfica para adicionar imagens, texturas e macros para o jogo.
\item[Flash Player ] É a ferramenta da adobe que executa projetos
feitos com o Flash
\item[FPS ] Frames por segundo, é a taxa de atualizacão que um jogo
consegue atingir. O que essa taxa mostra é quantas vezes o jogo
consegue fazer atualizacões na tela em um segundo. A quantidade padrão
utilizada para jogos é de 60 frames por segundo, ou seja, 60 fps.
\item[HTML ] HyperText Markup Language, a linguagem de marcação
utilizada para descrever as páginas de internet atualmente.
\item[HTTP ] HyperText Transfer Protocol, ou protocolo de
transferencia de hypertexto, é uma das principais tecnologias
utilizadas na internet. Esse protocolo é utilizado para enviar páginas
entre um servidor e um cliente.
\item[Inkscape ] É um editor de gráficos vetoriais de código aberto
utilizado para fazer desenhos vetoriais.
\item[IO ] Input/Output, entrada e saída, ou seja, trafego de dados.
\item[Java Applets ] Um Applet é um programa escrito em Java que pode
ser incluído em uma página HTML.Quando um navegador com suporte a
tecnologia Java acessa uma página com um applet, o código Java é
transferido para o computador do usuário e executador pela JVM do
navegador.
\item[JVM ] Java Virtual Machine, ou maquina virtual do Java, é a
maquina virtual que executa os softwares pré-compilados, escritos em
Java.
\item[plugin ] É um software externo que complementa outro software já
instalado.
\item[RIA ] Rich Internet Applications, ou aplicações ricas para
internet, são aplicações que possuem muito conteúdo multimídia e que
dão uma sensação diferente ao usuário, não parecem uma página web
comum.
\item[Sprite ] É uma imagem que define um conjunto de imagens que
serão utilizadas dentro do jogo. Essa imagem é mapeada para que apenas
mudando as posições dentro do jogo outra imagem possa ser mostrada,
dando assim a sensação de movimento ao jogador.
\item[TCP ] Transmission Control Protocol, ou Protocolo de Controle de
Transmissão, é uma dos protocolos mais utilizados em redes de
computadores, e seu trabalho é garantir que o pacote vai chegar até o
seu destino.
\item[WYSIWYG ] What You See Is What You Get, ou O que você vê é o que
você obtem. Significa a capacidade de um programa de computador de
permitir que um documento, enquanto manipulado na tela, tenha a mesma
aparência de sua utilização, usualmente sendo considerada final a
forma impressa.
\item[W3C ] World Wide Web Consortium, ou o Consórcio World Wide Web,
é um consórcio internacional, no qual seus filiados (empresas e
pessoas físicas) trabalham juntos para desenvolver padrões para a
internet.
\item[XHTML ] eXtensible Hypertext Markup Language.
\item[XML ] eXtensible Markup Language.

\end{description}
\newpage
